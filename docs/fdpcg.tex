\documentclass{article}
\usepackage{graphicx}
\usepackage{natbib}
\usepackage{graphicx}
\bibliographystyle{plainnat}
\usepackage{fancyheadings}
\usepackage{tabularx}
\usepackage{alltt, parskip, boxedminipage}
\usepackage{makeidx, multirow, longtable, tocbibind, amssymb}
%\usepackage{fullpage}
\makeindex
\usepackage[usenames]{color}
\definecolor{darkblue}{rgb}{0,0.05,0.35}

\usepackage[dvips, pagebackref, pdftitle={}, pdfcreator={epydoc 2.1}, bookmarks=true, bookmarksopen=false, pdfpagemode=UseOutlines, colorlinks=true, linkcolor=black, anchorcolor=black, citecolor=black, filecolor=black, menucolor=black, pagecolor=black, urlcolor=darkblue]{hyperref}
\setlength{\textheight}{21.5cm}
\setlength{\textwidth}{18cm}
\setlength{\hoffset}{-3.0cm}
\setlength{\footskip}{1.5cm}
\setlength{\headsep}{2.5cm}
\setlength{\voffset}{-2.5cm}
\newlength{\BCL} % base class length, for base trees.

\usepackage{everyshi}
 \makeatletter
 \let\totalpages\relax
 \newcounter{mypage}
 \EveryShipout{\stepcounter{mypage}}
 \AtEndDocument{\clearpage
    \immediate\write\@auxout{%
     \string\gdef\string\totalpages{\themypage}}}
 \makeatother

\newcommand{\esofooter}{
\includegraphics[height=1cm]{durlogo.eps}
{\bf \large Centre for \AA dvanced Instrumentation}
}
\newcommand{\esoheaderl}{
\includegraphics[height=1cm]{durlogo.eps}
\begin{tabularx}{9cm}{c}
\Large {\bf \esotitle} \\ \large {\bf(\esodoctype)}\\
\rightmark\hspace{0.1cm}
\end{tabularx}
\vfill
}
\newcommand{\esoheaderc}{
}
\newcommand{\esoheaderr}{
\begin{tabular}{|l|l|}\hline
Doc. number: & \esodocno \\ \hline
Release date: & \esoreleasedate \\ \hline
Issue number: & \esoissue \\ \hline
Page number: & Page \thepage \ of \totalpages \\ \hline
Author(s) & \esoauthorname \\ \hline
\end{tabular}


}

\pagestyle{fancy}
\cfoot[]{}
\lfoot[\esofooter]{\esofooter}
\lhead[\esoheaderl]{\esoheaderl}
\chead[\esoheaderc]{\esoheaderc}
\rhead[\esoheaderr]{\esoheaderr}
\renewcommand{\sectionmark}[1]{\markboth{#1}{#1}}

\newenvironment{Ventry}[1]%
  {\begin{list}{}{%
    \renewcommand{\makelabel}[1]{\texttt{##1:}\hfil}%
    \settowidth{\labelwidth}{\texttt{#1:}}%
    \setlength{\leftmargin}{\labelsep}%
    \addtolength{\leftmargin}{\labelwidth}}}%
  {\end{list}}


\begin{document}
\include{title}
\renewcommand{\esotitle}{FDPCG reconstruction}
\renewcommand{\esodocno}{AOSIM-PCG-UoD-001}
\renewcommand{\releasedate}{070329}
\thispagestyle{empty}
%This next command provides the CVS tag.  If you want a cvs tag on
%your document, add the following line at the start of the document,
%after replacing the & signs with dollar signs...
%\newcommand{\cvsID}{& &Id& (CVS)&}
\providecommand{\cvsID}{CVS ID not provided: document made on \today}

\begin{center}
\includegraphics{durlogo.eps}
\end{center}
\vspace{0.5cm}
\Huge
\begin{center}
\esoproject\\
\end{center}
\Large
\vspace{1cm}


{\bf 
\begin{tabular}{ll}
Document title: & \esotitle \vspace{0.5cm}\\ 

Documentation number: & \esodocno \vspace{0.5cm}\\ 

Document type: & \esodoctype \vspace{0.5cm}\\ 

Issue number:& \esoissue \vspace{0.5cm}\\ 

Release date: & \esoreleasedate \\ 

\end{tabular}
}

\normalsize
\vfill

\begin{tabular}{|l|l|l|p{5cm}|}
\hline
Document & \esoauthorname & Signature &\\
prepared by & \esoauthortype & and date &\\ \hline
Document & \esoapprovername & Signature &\\
approved by & \esoapprovertype & and date &\\ \hline
Document & \esoreleasername & Signature &\\
released by & \esoreleasertype & and date &\\ \hline
Document & \esoreviewername & Signature &\\
reviewed by & \esoreviewertype & and date &\\ \hline
\end{tabular}

\small
%\begin{alltt}
\cvsID
%\end{alltt}
\normalsize
%\lfoot[\esofooter]{\esofooter}
%\lhead[\esoheaderl]{\esoheaderl}
%\chead[\esoheaderc]{\esoheaderc}
%\rhead[\esoheaderr]{\esoheaderr}
%\renewcommand{\sectionmark}[1]{\markboth{#1}{#1}}

\pagebreak



\begin{center}
\Large
{\bf Change record\\ \vspace{1cm}}
\normalsize
\esochangerecord
\end{center}
\vspace{2cm}

\begin{center}
\Large
{\bf Notification list\\ \vspace{1cm}}
\normalsize
\esonotificationlist
\end{center}

\pagebreak

\begin{center}
\Large
{\bf Acronyms and abbreviations\\ \vspace{1cm}}
\normalsize
\esoabbreviations
\end{center}

\pagebreak

\begin{center}
\Large
{\bf Applicable documents\\ \vspace{1cm}}
\normalsize

\esoapplicabledocs
\end{center}
\vspace{2cm}

\begin{center}
\Large
{\bf Reference documents \\ \vspace{1cm}}
\normalsize

\esorefdocs
\end{center}

\pagebreak
\tableofcontents
\pagebreak

\section{Introduction}
This document gives an overview of the Fourier domain preconditioned
gradient (FDPCG) reconstructor, which is part of the AO simulation.
FDPCG is a wavefront reconstruction technique which at worst is $O(n
\log n)$ where $n$ is the number of actuators in the DM.  This should
be contrasted with conventional matrix-vector multiplication methods
which are $O(n^2)$, with an $O(n^3)$ operation to create the invert
the interaction matrix initially (which can be prohivitive for large
systems - days).  FDPCG is a minimum variance reconstructor (compared
with least squares using SOR or standard MVM).  It is also a zonal
reconstructor, working with mirror modes that correspond to individual
actuators being poked, rather than Zernikes being placed on the
mirror.  It may be possible to get it working for a modal system,
though this is unlikely.  It should be noted that SOR reconstruction
is $O(n)$, though performs badly at low SNR (actually, SOR appears to
perform better than FDPCG at low light levels, though is not always
desirable because it does not involve a poke matrix).

The FDPCG algorithm is complicated, and sensitive to parameters.
However, it currently works well in classical AO mode using the
\texttt{xinterp\_recon} reconstructor and suitable parameters in the
parameter file.  This document attempts to explain how it works, and
what sensible parameters should be set.

\section{What is it and how it works}
Preconditioned gradient methods are an iterative way of solving matrix
problems that can be written as $Ax=b$ where $x$ is unknown (and $A$
and $b$ are known.  It solves this problem without inverting the
matrix (which is an $O(n^3)$ operation at best.  It basically applies
an initial guess, computes a gradient for improvement, and then
improves on this.  The preconditioner in this case is in the Fourier
domain, and is chosen because of suitable structure of the matricees
(phase covariance, poke matrix etc).  Further information about this
algorithm can be found in \citet{2006ApOpt..45.5281Y} and associated
papers. 

This algorithm relies extensively on sparse matrix approximations, and
it is getting these correct that is part of the problem.  The phase
covariance matrix for mirror actuator locations must be computed.
This should then be approximated to a block circulant matrix with
circulant blocks.  The inverse of this matrix is then used, and a good
approximation is that this is then diagonal (otherwise a full matrix
multiplication needs to be made).  It is important to avoid any matrix-vector
multiplications as these are $O(n^2)$.  All the elements on the
diagonal are identical, and so the inverse of the phase covariance
matrix is reduced to a single value!  The noise covariance matrix
(which is usually diagonal anyway) can also be inverted in this way.

The interaction (poke) matrix can also be simplified.  For zonal
systems this is usually highly sparse anyway (our FDPCG implementation
assumes that it is).  We simplify it to eight diagonals, i.e. a
$8\times n$ dimension matrix.

The iterative nature of FDPCG is not ideal.  However, convergence is
fast, and typically only 2 or 3 iterations are sufficient for a
reasonable solution.  In our implementation, a minimum and maximum
iteration count are specified, along with a tolerance.  Reconstruction
is complete when the tolerance is met, or the maximum iteration count
reached (which ever is sooner).  The tolerance is the maximum
difference between results from the previous and current iteration.  

There are a number of options that can be used to tweak PCG.  However,
the default options (in the options dictionary) should give best
results, and so are not discussed further.  

\section{Extension to MCAO and other advanced simulations}
If the FDPCG algorithm needs to be expanded to other simulations, more
work will be required.  The phase covariance matrix structure will
need examining.  This will then lead to a block diagonal matrix with
blocks that are BCCB.  Whether there is a simple way to invert this
should then be investigated.  A simple structure for the poke matrix
will also be required.  The algorithm will then need to be tested to
check that it behaves correctly with more than one atmospheric layer
(I think it generally has been written to do this, but not tested).
All should then work (hopefully!).


\bibliography{references}
\printindex
\end{document}
