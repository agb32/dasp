\newcommand{\cvsID}{$ $Id: wfsPipe.tex,v 1.5 2006/07/05 13:02:28 ali Exp $ (CVS)$}
\documentclass{article}
\usepackage{tabularx}
\newcommand{\mod}[1]{\texttt{#1}}
\usepackage{natbib}
\usepackage{graphicx}
\bibliographystyle{plainnat}
\usepackage{fancyheadings}
\usepackage{tabularx}
\usepackage{alltt, parskip, boxedminipage}
\usepackage{makeidx, multirow, longtable, tocbibind, amssymb}
%\usepackage{fullpage}
\makeindex
\usepackage[usenames]{color}
\definecolor{darkblue}{rgb}{0,0.05,0.35}

\usepackage[dvips, pagebackref, pdftitle={}, pdfcreator={epydoc 2.1}, bookmarks=true, bookmarksopen=false, pdfpagemode=UseOutlines, colorlinks=true, linkcolor=black, anchorcolor=black, citecolor=black, filecolor=black, menucolor=black, pagecolor=black, urlcolor=darkblue]{hyperref}
\setlength{\textheight}{21.5cm}
\setlength{\textwidth}{18cm}
\setlength{\hoffset}{-3.0cm}
\setlength{\footskip}{1.5cm}
\setlength{\headsep}{2.5cm}
\setlength{\voffset}{-2.5cm}
\newlength{\BCL} % base class length, for base trees.

\usepackage{everyshi}
 \makeatletter
 \let\totalpages\relax
 \newcounter{mypage}
 \EveryShipout{\stepcounter{mypage}}
 \AtEndDocument{\clearpage
    \immediate\write\@auxout{%
     \string\gdef\string\totalpages{\themypage}}}
 \makeatother

\newcommand{\esofooter}{
\includegraphics[height=1cm]{durlogo.eps}
{\bf \large Centre for \AA dvanced Instrumentation}
}
\newcommand{\esoheaderl}{
\includegraphics[height=1cm]{durlogo.eps}
\begin{tabularx}{9cm}{c}
\Large {\bf \esotitle} \\ \large {\bf(\esodoctype)}\\
\rightmark\hspace{0.1cm}
\end{tabularx}
\vfill
}
\newcommand{\esoheaderc}{
}
\newcommand{\esoheaderr}{
\begin{tabular}{|l|l|}\hline
Doc. number: & \esodocno \\ \hline
Release date: & \esoreleasedate \\ \hline
Issue number: & \esoissue \\ \hline
Page number: & Page \thepage \ of \totalpages \\ \hline
Author(s) & \esoauthorname \\ \hline
\end{tabular}


}

\pagestyle{fancy}
\cfoot[]{}
\lfoot[\esofooter]{\esofooter}
\lhead[\esoheaderl]{\esoheaderl}
\chead[\esoheaderc]{\esoheaderc}
\rhead[\esoheaderr]{\esoheaderr}
\renewcommand{\sectionmark}[1]{\markboth{#1}{#1}}

\newenvironment{Ventry}[1]%
  {\begin{list}{}{%
    \renewcommand{\makelabel}[1]{\texttt{##1:}\hfil}%
    \settowidth{\labelwidth}{\texttt{#1:}}%
    \setlength{\leftmargin}{\labelsep}%
    \addtolength{\leftmargin}{\labelwidth}}}%
  {\end{list}}


\begin{document}
\include{wfsPipetitle}
\thispagestyle{empty}
%This next command provides the CVS tag.  If you want a cvs tag on
%your document, add the following line at the start of the document,
%after replacing the & signs with dollar signs...
%\newcommand{\cvsID}{& &Id& (CVS)&}
\providecommand{\cvsID}{CVS ID not provided: document made on \today}

\begin{center}
\includegraphics{durlogo.eps}
\end{center}
\vspace{0.5cm}
\Huge
\begin{center}
\esoproject\\
\end{center}
\Large
\vspace{1cm}


{\bf 
\begin{tabular}{ll}
Document title: & \esotitle \vspace{0.5cm}\\ 

Documentation number: & \esodocno \vspace{0.5cm}\\ 

Document type: & \esodoctype \vspace{0.5cm}\\ 

Issue number:& \esoissue \vspace{0.5cm}\\ 

Release date: & \esoreleasedate \\ 

\end{tabular}
}

\normalsize
\vfill

\begin{tabular}{|l|l|l|p{5cm}|}
\hline
Document & \esoauthorname & Signature &\\
prepared by & \esoauthortype & and date &\\ \hline
Document & \esoapprovername & Signature &\\
approved by & \esoapprovertype & and date &\\ \hline
Document & \esoreleasername & Signature &\\
released by & \esoreleasertype & and date &\\ \hline
Document & \esoreviewername & Signature &\\
reviewed by & \esoreviewertype & and date &\\ \hline
\end{tabular}

\small
%\begin{alltt}
\cvsID
%\end{alltt}
\normalsize
%\lfoot[\esofooter]{\esofooter}
%\lhead[\esoheaderl]{\esoheaderl}
%\chead[\esoheaderc]{\esoheaderc}
%\rhead[\esoheaderr]{\esoheaderr}
%\renewcommand{\sectionmark}[1]{\markboth{#1}{#1}}

\pagebreak



\begin{center}
\Large
{\bf Change record\\ \vspace{1cm}}
\normalsize
\esochangerecord
\end{center}
\vspace{2cm}

\begin{center}
\Large
{\bf Notification list\\ \vspace{1cm}}
\normalsize
\esonotificationlist
\end{center}

\pagebreak

\begin{center}
\Large
{\bf Acronyms and abbreviations\\ \vspace{1cm}}
\normalsize
\esoabbreviations
\end{center}

\pagebreak

\begin{center}
\Large
{\bf Applicable documents\\ \vspace{1cm}}
\normalsize

\esoapplicabledocs
\end{center}
\vspace{2cm}

\begin{center}
\Large
{\bf Reference documents \\ \vspace{1cm}}
\normalsize

\esorefdocs
\end{center}

\pagebreak
\tableofcontents
\pagebreak

\section{Scope}
This document describes the details of the FPGA acceleration available
with the AO simulation software package which is funded by the rolling
grant.

This document is relevant to anyone who is involved with the
development of the AO simulation software.  For a brief description on
how to set up your own AO simulation, please read \citet{dummies}.
This document can also be used by FPGA programmers who which to alter
the pipeline.

\subsection{Overview}

The AO simulation package is to be used to simulate the effect of
atmospheric turbulence on large astronomical telescopes.  It has been
designed in such a fashion that it is modular, and so extensible
should future requirements change.  Each module is designed to perform
a specific simulation task, and are connected together by the user in
such a way that coherent results will be obtained from the simulation.

These modules communicate with each other by passing data, either
within their process, using shared memory, or using MPI message
passing.  The simulation framework has been designed so that this
method of message passing is more or less transparent to the user,
with the exception that the user should place the required
communication protocol modules in the correct place.

Additionally, when running on a Cray XD1 with FPGA application
acceleration processors installed, these processors can be used to
implement a wavefront sensing (WFS) pipeline, which can operate at
over 600 times faster than the equivalent software version.  Details
of this WFS pipeline are provided here.
\section{Making use of the hardware wavefront sensing pipeline}
To make use of the hardware wavefront sensing pipeline, import the
wfscent.py module into your simulation, and create one or more
wfscent.wfscent objects.  These will read the provided simulation
configuration file to determine whether the FPGAs can and should be
used, and will then initialise as required.  If parameters in this
file are incompatible with FPGA use, warning messages will be printed
and the FPGAs will not be used, for example if the high light level
Shack-Hartmann images are required to be a size incompatible with that
available in the FPGA.  

Examples using the wfscent module can be found in the aosim CVS
repository in the examples/fpga/ directory.  The simulation
configuration file should contain entries for initFPGA (whether the
FPGA should be initialised), useFPGA (whether the FPGA should be used
by default, as you may wish to initialise it only), FPGAWFSBitFile
(the filename of the FPGA binary, typically
aosim/fpga/wfspipe.bin.ufp) and waitFPGA (whether to wait for the FPGA
computation to complete before continuing to another module).  If the
waitFPGA flag is set, the wfscent module will simply sit waiting for
the centroid data to be computed, and this will waste CPU time.  If
you know you have another process that can be active whilst this is
going on (for example, computation of science object statistics) then
set this flag to zero.  However, if the reconstructor object is called
directly after the wfscent object, it is probably best to set this
flag to one, since you have no guarantee that the new centroid values
will be ready before being used by the reconstructor otherwise.
\subsection{Turbo button}
While a simulation is running, the FPGA can be switched on or off,
rather like a turbo button.  This can be controlled from the simCtrl
GUI, and an example of how to do this is seen in the
aosim/example/fpga/ directory.  When the FPGA is switched off, the
conventional software algorithm will be used.

\section{The hardware wavefront sensing pipeline}
The FPGA based WFS pipeline accepts the telescope pupil phase as an
input (32 bit floating point), and provides centroid values from
simulated Shack-Hartmann arrays as an output, again as 32 bit floating
point.  The pipeline operates as a (configurable) black box, and it is
not possible to view any intermediate state, such as Shack-Hartmann images.

The pipeline performs the following operations, with each of these
coded as a separate VHDL module:
\begin{enumerate}
\item Converts the floating point phase input into a fixed point value,
  modulo 2$\pi$ (17 bit fixed point in 1.16 format).  
\item Adds a tilt vector to each sub-aperture, again modulo 2$\pi$,
  giving 17 bit fixed point values.
\item Calculates the sine and cosine values of each fixed point phase
  value, giving a 16 bit fixed point output (0.16 format).
\item Applies a pupil map to the sub-apertures, returning zero if the
  pixel is masked out, or the sine and cosine values if the pixel will
  have light falling on it.
\item A 2D FFT is computed for each sub-aperture, inputs being the
  sine and cosine values, and outputs being 28 bits wide (in butterfly
  order).
\item The 2D FFT output is rearranged into an order such that the DC
  term will fall at the centre of the sub-aperture, i.e.\ the maximum
  intensity for a flat input phase will fall here.
\item The square modulus of the signal is then computed (real and
  imaginary parts are squared and added), to create the high light level
  image.  The signal is now 55 bits wide.
\item The data is rescaled into a floating point format with a 22 bit
  wide mantissa, and a 6 bit wide exponent.  This is not IEEE
  compatible, but is ideal for later operations.
\item High light level images are integrated as required (up to 63
  frames can be integrated together).  The output is floating point
  format with 22 bit mantissa and 6 bit exponent (no sign bit).
\item The integrated high light level images are binned together, if
  required.  This allows the sub-aperture size to be reduced to
  anything from ($2\times2$) up to the FFT size ($8\times8$,
  $16\times16$ or $32\times32$).  Interpolation is performed as
  required.  The output is again floating point format with a 22 bit
  wide mantissa.
\item The sub-apertures are normalised, i.e.\ the value in each pixel
  is divided by the total light in the sub-aperture, and multiplied by
  a user defined value, equal to the star intensity.  A 23 bit output
  in 16.7 fixed point format is provided, representing the mean number
  of photons falling on each pixel of the detector.
\item A sky background is added to each pixel.  This is a constant
  value.
\item Random photon shot noise is introduced to each pixel, dependent
  on the mean light level in this pixel.  The output from this stage
  is 16 bit integer, representing the number of photo-electrons
  created in the semi-conductor detector.  Overflow is represented by
  all 16 bits being set.
\item Random CCD readout noise is introduced to each pixel, with a
  given mean and standard deviation.  The output is 16 bits, with
  overflow represented by all bits set.
\item A threshold value is applied to the signal, with everything
  below this value being set to zero, and this value being subtracted
  from everything else (except overflow pixels).  The output is 16
  bits.
\item A centroiding algorithm is performed for each sub-aperture,
  returning the x and y centroid values in 32 bit format.
\item The centroid values are converted to IEEE standard 32 bit
  floating point format, and returned to the host memory.


\end{enumerate}

\subsection{Pupil map}
The pupil map is stored in a bit fashion in one external 4~MB SRAM.
By use of two or four fold symmetry (user selectable), WFS detectors
with up to 32 million pixels can be simulated, enough for all but the
largest cases.  The maximum size of a pupil is 1023 sub-apertures in
each dimension, and minimum is one sub-aperture.  If larger numbers of
sub-apertures are required, they can be used, but the pupil function
will not perform correctly.  In this case, it is best to set all the
pupil function to one (every pixel valid), and ignore bad centroid
values at the reconstruction stage.

\subsection{FFT}
The 2D FFT can be 8, 16 or 32 pixels wide in each dimension, user
selectable.  Overflow cannot occur since the output is wide enough to
ensure it does not.  The input phase values for each sub-aperture can
be between ($2\times2$) to ($32\times32$) values in size.  The FFT
dimensions must be equal to or larger than this, with zero padding of
the data occurring automatically if larger.

\subsection{Integrate}
The number of iterations to integrate for is user selectable.  When
integrating, sub-apertures that are adjacent in host CPU memory are
added (high light level images are added).  This means that the phase
data must be interleaved for the different iterations, but means that
large amounts of storage are not needed on the FPGA.  Up to 63
iterations can be integrated.

\subsection{Bin}
The binning of high light level images is user selectable, and
different x and y bin sizes can be chosen, for example you could start
with a ($24\times24$) phase array for each sub-aperture, zero pad it
(inside the FPGA) to do a ($32\times32$) FFT and then bin the image
down to ($10\times13$) pixels on the (simulated) CCD detector.
It is only possible to bin down, so you cannot go from a
($16\times16$) FFT to a ($24\times24$) binned image.  The minimum
allowed size is ($2\times2$) pixels and maximum is ($32\times32$)
pixels (the maximum FFT size).

\subsection{Normalise}
Normalisation of the high light level images is carried out using a
user definable value representing the total intensity in a
sub-aperture.  This scaling factor is provided in 20.7 fixed point
format, equal to the total number of photons in the sub-aperture.

\subsection{Sky brightness}
A sky brightness value can be added, in 16.7 fixed point format, the
number of photons to be added to each pixel being user selectable.

\subsection{Photon shot noise}
Photon shot noise is added to the signal using pre-generated random
numbers stored in FPGA external memory (4~MBits).  The value of the
input signal is used to select a bin.  A memory location corresponding
to the bin is then read, giving an address of the next random variable
for this light level to access.  The bins are closely spaced for the
lowest light levels, with sparser spacing as light level increases.
For light levels greater than 32 photons per pixel, the central limit
theorem is applied to give a Gaussian random variable with mean and
variance equal to the mean light level.  This provides a realistic
Poisson distribution, which would otherwise be difficult to generate
within the FPGA on a cycle by cycle basis.

\subsection{Readout noise}
A Mersenne Twister algorithm is used to generate uniform random
numbers.  Eight 8-byte random variables are then added together giving
a Gaussian distribution with a mean of 127 and standard deviation of
26.11.  This distribution is then scaled to provide a random Gaussian
variable with a mean and variance chosen by the user.  This signal is
then added to the CCD signal.  If you want to add readout noise with a
mean of $m$ and standard deviation of $s$, the values given to the
FPGA will be $m-s*127/26.11$ in 16.8 fixed point format and $s/26.11$
in 8.8 fixed point format.

\subsection{Threshold}
A 16 bit threshold value can be selected by the user.  This value is
subtracted from each signal, and every negative number is then set to
zero.  

\subsection{Centroid}
The size of the sub-apertures at centroid stage are determined by the
size of sub-apertures after the binning operation.  A standard
centroid algorithm is used.  The maximum size of sub-apertures is 32
pixels in one direction, and the minimum is 2 pixels in one direction.

\subsection{Calibration sources}
A calibration source can be used by setting CCD read noise and
deviation to be zero, and by initialising the photon shot noise random
variable arrays to equal the number of photons rounded to the nearest
integer, instead of being a random Poisson distribution for each mean
light level.  The wfscent.py module shows how to do this.

\section{Further information}
\subsection{Input array order}
The input data is a 32 bit floating point phase array with dimensions
equal to (N$_{sx}$, N$_{sy}$, N$_{i}$, N$_p$, N$_p$) where N$_{sx}$ is
the number of sub-apertures in the x direction, N$_{sy}$ is the number
of sub-apertures in the y direction, N$_i$ is the number of iterations
to integrate over and N$_p$ is the x and y dimensions of each phase
input (between 2-32).

\subsection{Output array order}
The centroid values are returned as 32 bit floating point numbers in
an array with dimensions (N$_{sx}$, N$_{sy}$, 2) where [:,:,0] are the x
centroid locations and [:,:,1] are the y centroid locations.

\subsection{Register addresses}
The addresses of registers in the FPGA, and their use is given in 
table~\ref{table:reg}.  These registers are set using the
\texttt{fpga.writeReg(fpid,VALUE,ADDRESS)} function.

\begin{table}
\caption{A table showing register addresses for configuration values}
\label{table:reg}
\begin{tabularx}{\linewidth}{lXXX}
Address & Function &Description& Users\\ \hline
512 & Pipe select &Determines where data should go&wfsPipe.vhd\\
519 & Number of integrations &&integrate.vhd, pupilfn.vhd\\
520 & x subap size & Number of x pixels & centroid16singleInput.vhd\\
521 & y subap size & Numer of y pixels & centroid16singleInput.vhd\\
522 & FFT size & 8, 16 or 32 & fft2d.vhd, fliparray.vhd, scale.vhd,
bin.vhd\\
523 & phase size & 2-32 & fft2d.vhd, tilt.vhd, pupilfn.vhd\\
524 & source intensity & of guide star & normalise.vhd\\
525 & QDR address & Current QDR location & fillQdrMem.vhd\\
526 & 1/fft size*$2^{15}$ &&tilt.vhd\\
528 & sky brightness && addconst.vhd\\
529 & threshold value && threshold.vhd\\
530 & readnoise mean & Appropriately scaled & readoutNoise.vhd\\
531 & readnoise RMS & Appropriately scaled & readoutNoise.vhd\\
532 & x subap size & Same as 520 & bin.vhd, normalise.vhd\\
533 & y subap size & Same as 521 & bin.vhd, normalise.vhd\\
534 & Number of x and y subaps & nx\&(ny<<10)&pupilfn.vhd\\
535 & pupil symmetry type & 0, 1 or 2 & pupilfn.vhd\\
536 & Bin configuration & See wfscent.py & bin.vhd\\
... &&&\\
543 & Last bin configuration &&bin.vhd\\ 
\hline
\end{tabularx}
\end{table}
\subsection{Operation}
To operate the WFS pipeline in the FPGA, it is necessary to set the
WFS configuration registers, initialise memory for the photon shot
noise algorithm, memory for the pupil function, and memory for the
random number seed.  Once this is ready, it is then necessary to
initialise the pipeline registers which determine the DMA in and out
size and addresses, and start or stop the pipeline.  Further details
can be found in the wfscent.py module.

\subsection{Pupil masks}
When a pupil mask is used, you may still get non-zero centroid values
for parts of the pupil receiving no light.  This is because of the CCD
readout noise, and is realistic for a given WFS.  However, these
centroid values should then be ignored for the reconstructor.

\pagebreak
\bibliography{references}


\printindex
\end{document}
