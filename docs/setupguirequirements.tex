\documentclass{article}
\usepackage{natbib}
\usepackage{graphicx}
\bibliographystyle{plainnat}
\usepackage{fancyheadings}
\usepackage{tabularx}
\usepackage{alltt, parskip, boxedminipage}
\usepackage{makeidx, multirow, longtable, tocbibind, amssymb}
%\usepackage{fullpage}
\makeindex
\usepackage[usenames]{color}
\definecolor{darkblue}{rgb}{0,0.05,0.35}

\usepackage[dvips, pagebackref, pdftitle={}, pdfcreator={epydoc 2.1}, bookmarks=true, bookmarksopen=false, pdfpagemode=UseOutlines, colorlinks=true, linkcolor=black, anchorcolor=black, citecolor=black, filecolor=black, menucolor=black, pagecolor=black, urlcolor=darkblue]{hyperref}
\setlength{\textheight}{21.5cm}
\setlength{\textwidth}{18cm}
\setlength{\hoffset}{-3.0cm}
\setlength{\footskip}{1.5cm}
\setlength{\headsep}{2.5cm}
\setlength{\voffset}{-2.5cm}
\newlength{\BCL} % base class length, for base trees.

\usepackage{everyshi}
 \makeatletter
 \let\totalpages\relax
 \newcounter{mypage}
 \EveryShipout{\stepcounter{mypage}}
 \AtEndDocument{\clearpage
    \immediate\write\@auxout{%
     \string\gdef\string\totalpages{\themypage}}}
 \makeatother

\newcommand{\esofooter}{
\includegraphics[height=1cm]{durlogo.eps}
{\bf \large Centre for \AA dvanced Instrumentation}
}
\newcommand{\esoheaderl}{
\includegraphics[height=1cm]{durlogo.eps}
\begin{tabularx}{9cm}{c}
\Large {\bf \esotitle} \\ \large {\bf(\esodoctype)}\\
\rightmark\hspace{0.1cm}
\end{tabularx}
\vfill
}
\newcommand{\esoheaderc}{
}
\newcommand{\esoheaderr}{
\begin{tabular}{|l|l|}\hline
Doc. number: & \esodocno \\ \hline
Release date: & \esoreleasedate \\ \hline
Issue number: & \esoissue \\ \hline
Page number: & Page \thepage \ of \totalpages \\ \hline
Author(s) & \esoauthorname \\ \hline
\end{tabular}


}

\pagestyle{fancy}
\cfoot[]{}
\lfoot[\esofooter]{\esofooter}
\lhead[\esoheaderl]{\esoheaderl}
\chead[\esoheaderc]{\esoheaderc}
\rhead[\esoheaderr]{\esoheaderr}
\renewcommand{\sectionmark}[1]{\markboth{#1}{#1}}

\newenvironment{Ventry}[1]%
  {\begin{list}{}{%
    \renewcommand{\makelabel}[1]{\texttt{##1:}\hfil}%
    \settowidth{\labelwidth}{\texttt{#1:}}%
    \setlength{\leftmargin}{\labelsep}%
    \addtolength{\leftmargin}{\labelwidth}}}%
  {\end{list}}

\newcommand{\mod}[1]{\texttt{#1}}
\begin{document}
\include{setupguirequirementstitle}
\thispagestyle{empty}
%This next command provides the CVS tag.  If you want a cvs tag on
%your document, add the following line at the start of the document,
%after replacing the & signs with dollar signs...
%\newcommand{\cvsID}{& &Id& (CVS)&}
\providecommand{\cvsID}{CVS ID not provided: document made on \today}

\begin{center}
\includegraphics{durlogo.eps}
\end{center}
\vspace{0.5cm}
\Huge
\begin{center}
\esoproject\\
\end{center}
\Large
\vspace{1cm}


{\bf 
\begin{tabular}{ll}
Document title: & \esotitle \vspace{0.5cm}\\ 

Documentation number: & \esodocno \vspace{0.5cm}\\ 

Document type: & \esodoctype \vspace{0.5cm}\\ 

Issue number:& \esoissue \vspace{0.5cm}\\ 

Release date: & \esoreleasedate \\ 

\end{tabular}
}

\normalsize
\vfill

\begin{tabular}{|l|l|l|p{5cm}|}
\hline
Document & \esoauthorname & Signature &\\
prepared by & \esoauthortype & and date &\\ \hline
Document & \esoapprovername & Signature &\\
approved by & \esoapprovertype & and date &\\ \hline
Document & \esoreleasername & Signature &\\
released by & \esoreleasertype & and date &\\ \hline
Document & \esoreviewername & Signature &\\
reviewed by & \esoreviewertype & and date &\\ \hline
\end{tabular}

\small
%\begin{alltt}
\cvsID
%\end{alltt}
\normalsize
%\lfoot[\esofooter]{\esofooter}
%\lhead[\esoheaderl]{\esoheaderl}
%\chead[\esoheaderc]{\esoheaderc}
%\rhead[\esoheaderr]{\esoheaderr}
%\renewcommand{\sectionmark}[1]{\markboth{#1}{#1}}

\pagebreak



\begin{center}
\Large
{\bf Change record\\ \vspace{1cm}}
\normalsize
\esochangerecord
\end{center}
\vspace{2cm}

\begin{center}
\Large
{\bf Notification list\\ \vspace{1cm}}
\normalsize
\esonotificationlist
\end{center}

\pagebreak

\begin{center}
\Large
{\bf Acronyms and abbreviations\\ \vspace{1cm}}
\normalsize
\esoabbreviations
\end{center}

\pagebreak

\begin{center}
\Large
{\bf Applicable documents\\ \vspace{1cm}}
\normalsize

\esoapplicabledocs
\end{center}
\vspace{2cm}

\begin{center}
\Large
{\bf Reference documents \\ \vspace{1cm}}
\normalsize

\esorefdocs
\end{center}

\pagebreak
\tableofcontents
\pagebreak


\section{Introduction}
This guide has been created to provide information about requirements
for a simulation setup GUI.  For information about the simulation
please see \citet{dummies}.

\section{The rough idea}
The rough idea is to have a point and click interface which will allow
the user to put together a simulation by choosing modules, placing
them, linking them together in the correct order, deciding on a
parallelisation strategy, and then running the simulation, in a
similar (but better and more flexible) way that CAOS \citep{caos}
works.

\section{Requirements}
As I see it, we will require something that will allow the user to
select a module from a list of predefined (previously used) modules.
This module will then be placed on a canvas, and the user can move it
around as suits.  Additionally, new modules can be added to the list
of available ones (e.g.\ with a file open dialogue, allowing the user
to select a python file containing the module).

The outputs of modules on the canvas can be linked to inputs of other
modules, by selecting a ``link'' button (or similar).  There can be
more than one connection to each output and input.  Additionally, the
links should be able to be labelled (and should be automatically
labelled depending on the name of the module they come from).  

\subsection{Parallelisation}
An AO simulation can be parallelised to run on different processors.
At the moment, we should consider only running on the Cray nodes, but
allow for future use on different platforms.  Considering only the
Cray should have no effect on the design of the setup GUI.

The simulation framework has been written to make advantage of the
most efficient parallel data sharing methods.  Modules running on the
same processor are actually part of the same program.  Modules running
on a different processor, but with access to shared memory regions
(e.g.\ SMP machines), share data using these shared memory regions, to
avoid making unnecessary copies of the data.  Modules running on other
processors communicate via MPI.  So, the GUI requires something to
allow users to select which processors should run on the same
processor (i.e.\ be part of the same process), which should run on
different processors on the same node of the XD1 (i.e.\ can share
memory), and which should run on different nodes.  For this, I would
suggest (open to discussion/change) that there are a number of buttons
which represent the different processors.  The user can then select
one of these, and click on all the modules to be assigned to this
processor.  The user would then choose a different processor, and
assign different modules to this.  

It would be good to have some sort of visual display of where the
modules are assigned.  My suggested way of doing this would be to
assign a colour to each node, and use a different shade of this colour
for each processor within the node.  

\subsubsection{Default schemes}
For ease of use, the GUI should also be able to suggest a
parallelisation scheme depending on which modules are tied in what
way.  The user should then be able to click a ``parallelise'' button,
which would then automatically assign the modules to processors as
appropriate.

\subsubsection{Internals}
When the user selects the nodes for modules to run on, they aren't
really, but rather just specifying how the simulation file is set up.
It is only when the simulation is run (using mpirun) that the
processes are assigned to nodes.  So, a simulation created to run as
12 processes on the 6 Cray nodes (12 processors) can actually be run
on a single node.  Ask if confused!


\subsection{Writing the simulation files}
Once the user has set up the simulation in the GUI, they should be
able to click a ``build'' button (or similar) which causes the
simulation to be written to a simulation file.
Parameter definitions of the modules in this file can then be created
using the existing \texttt{paramgui} software.  When writing this
simulation file, the GUI will take care of issues regarding
parallelisation, i.e.\ it will place MPI and SHM modules at the
correct points, the user doesn't need to know about these.  This build
process will also create a short bash script which can be used to run
the simulation.  For example, this file might contain:

\texttt{mpirun -np 4 -hostlist n6-c437 n5-c437 n4-c437 n3-c437
  /usr/local/bin/mpipython /path/to/sim/file/simfile.py arguments}

An example of a fairly complicated simulation file is given in
projects/ali/gloa/glaorun\_nompi\_common.py.  However, this file
doesn't include any MPI or SHM connections, so for examples of these,
see examples/test2/newtest-scimpi.py for example.

\subsection{Additional information}
There are pieces of additional information that the user must be able
to specify.  These include:
\begin{enumerate}
\item Batch number (the simulation will be run with
  --batchno=BATCHNUMBER).
\item Arguments to be passed at runtime, e.g.\ --start-paused etc.
\item Initial commands (see projects/ali/glao/glaorun\_nompi\_common.py
  for an example).
\item Finalisation commands to be run after simulation has finished
  (e.g.\ to print results, see previous example).
\item Other things yet to be decided.
\end{enumerate}


\subsection{Modules}
When adding a module to the simulation, it should be possible to
define some properties about this module (e.g.\ a dialog box may
appear if a module is double clicked on).  These include a name
for the module (for example, there might be several wavefront sensing
modules in a given simulation, and they could then be named wfs\_1,
wfs\_2, etc).  There are other properties needed, but I can't
think of them at the moment.  

\subsection{Graphical control}
When a simulation is running, the user is able to connect to it, and
query or change it.  This is done using simgui.py.  As part of this,
it is necessary to create a XML parameter file which determines
default buttons available for simgui.  The simulation setup GUI should
therefore create default buttons (taken from the simulation modules),
and allow the user to alter and edit these, as well as adding other buttons.


\pagebreak
\bibliography{references}

\printindex
\end{document}
