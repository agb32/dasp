\documentclass{article}
\usepackage{natbib}
\usepackage{graphicx}
\bibliographystyle{plainnat}
\usepackage{fancyheadings}
\usepackage{tabularx}
\usepackage{alltt, parskip, boxedminipage}
\usepackage{makeidx, multirow, longtable, tocbibind, amssymb}
%\usepackage{fullpage}
\makeindex
\usepackage[usenames]{color}
\definecolor{darkblue}{rgb}{0,0.05,0.35}

\usepackage[dvips, pagebackref, pdftitle={}, pdfcreator={epydoc 2.1}, bookmarks=true, bookmarksopen=false, pdfpagemode=UseOutlines, colorlinks=true, linkcolor=black, anchorcolor=black, citecolor=black, filecolor=black, menucolor=black, pagecolor=black, urlcolor=darkblue]{hyperref}
\setlength{\textheight}{21.5cm}
\setlength{\textwidth}{18cm}
\setlength{\hoffset}{-3.0cm}
\setlength{\footskip}{1.5cm}
\setlength{\headsep}{0.5cm}
\setlength{\voffset}{-2.5cm}
\newlength{\BCL} % base class length, for base trees.

\usepackage{everyshi}
 \makeatletter
 \let\totalpages\relax
 \newcounter{mypage}
 \EveryShipout{\stepcounter{mypage}}
 \AtEndDocument{\clearpage
    \immediate\write\@auxout{%
     \string\gdef\string\totalpages{\themypage}}}
 \makeatother

\newcommand{\daspfooter}{
\includegraphics[height=1cm]{durlogo.eps}
{\bf \large Centre for \AA dvanced Instrumentation}
}
\newcommand{\daspheaderl}{
\includegraphics[height=1cm]{durlogo.eps}
\begin{tabularx}{9cm}{c}
\Large {\bf \dasptitle} \\ \large {\bf(\daspdoctype)}\\
\rightmark\hspace{0.1cm}
\end{tabularx}
\vfill
}
\newcommand{\daspheaderc}{
}
\newcommand{\daspheaderr}{
\begin{tabular}{|l|l|}\hline
Doc. number: & \daspdocno \\ \hline
Release date: & \daspreleasedate \\ \hline
Issue number: & \daspissue \\ \hline
Page number: & Page \thepage \ of \totalpages \\ \hline
Author(s) & \daspauthorname \\ \hline
\end{tabular}


}

\pagestyle{fancy}
\cfoot[]{}
\lfoot[\daspfooter]{\daspfooter}
\lhead[\daspheaderl]{\daspheaderl}
\chead[\daspheaderc]{\daspheaderc}
\rhead[\daspheaderr]{\daspheaderr}
\renewcommand{\sectionmark}[1]{\markboth{#1}{#1}}

\newenvironment{Ventry}[1]%
  {\begin{list}{}{%
    \renewcommand{\makelabel}[1]{\texttt{##1:}\hfil}%
    \settowidth{\labelwidth}{\texttt{#1:}}%
    \setlength{\leftmargin}{\labelsep}%
    \addtolength{\leftmargin}{\labelwidth}}}%
  {\end{list}}

\newcommand{\mod}[1]{\texttt{#1}}
\begin{document}
\newcommand{\esoproject}{AO Simulation Project}
\newcommand{\esotitle}{AO Simulation schema definition}
\newcommand{\esodocno}{AOSIM-SCH-UoD-001}
\newcommand{\esodoctype}{Internal}
\newcommand{\esoissue}{0.1.1}
\newcommand{\esoreleasedate}{\today}
\newcommand{\esoauthorname}{Alastair Basden}
\newcommand{\esoauthortype}{AO sim team member}
\newcommand{\esoapprovername}{Alastair Basden}
\newcommand{\esoapprovertype}{AO sim team member}
\newcommand{\esoreleasername}{Alastair Basden}
\newcommand{\esoreleasertype}{AO sim team member}
\newcommand{\esoreviewername}{Alastair Basden}
\newcommand{\esoreviewertype}{AO sim team member}
\newcommand{\esochangerecord}{
\begin{tabular}{|l|l|l|l|}
\hline
Issue number & Release date & section(s) affected & Description of
change/remarks\\ \hline
0.1.0 & 051111 & All & First draft \\ \hline
0.1.1 & 051111 & All & Corrections made to first draft\\ \hline
\end{tabular}
}
\newcommand{\esonotificationlist}{
Alastair Basden\\
Francois Assemat\\
Richard Wilson\\
Tim Morris\\
Tim Butterley\\
Ali Bharmal\\
}
\newcommand{\esoabbreviations}{
\begin{tabular}{rl}
AO & Adaptive Optics\\
ESO & European Southern Observatory\\
\end{tabular}
}
\newcommand{\esoapplicabledocs}{
\begin{tabular}{|l|l|l|}\hline
AD Number & Document title & Doc number/publication/location \\ \hline
AD02 & Simulation API document & AOSIM-API-UoD-001\\ \hline
\end{tabular}
}
\newcommand{\esorefdocs}{
\begin{tabular}{|l|l|l|}\hline
RD number & Document title & Doc number/publication/location \\ \hline
RD01 & Durham FPGA website & www.durham.ac.uk/rtcs.project \\ \hline
\end{tabular}
}
\title{\esotitle}



\thispagestyle{empty}
%This next command provides the CVS tag.  If you want a cvs tag on
%your document, add the following line at the start of the document,
%after replacing the & signs with dollar signs...
%\newcommand{\cvsID}{& &Id& (CVS)&}
\providecommand{\cvsID}{CVS ID not provided: document made on \today}

\begin{center}
\includegraphics{durlogo.eps}
\end{center}
\vspace{0.5cm}
\Huge
\begin{center}
\daspproject\\
\end{center}
\Large
\vspace{1cm}


{\bf 
\begin{tabular}{ll}
Document title: & \dasptitle \vspace{0.5cm}\\ 

Documentation number: & \daspdocno \vspace{0.5cm}\\ 

Document type: & \daspdoctype \vspace{0.5cm}\\ 

Issue number:& \daspissue \vspace{0.5cm}\\ 

Release date: & \daspreleasedate \\ 

\end{tabular}
}

\normalsize
\vfill

\begin{tabular}{|l|l|l|p{5cm}|}
\hline
Document & \daspauthorname & Signature &\\
prepared by & \daspauthortype & and date &\\ \hline
Document & \daspapprovername & Signature &\\
approved by & \daspapprovertype & and date &\\ \hline
Document & \daspreleasername & Signature &\\
released by & \daspreleasertype & and date &\\ \hline
Document & \daspreviewername & Signature &\\
reviewed by & \daspreviewertype & and date &\\ \hline
\end{tabular}

\small
%\begin{alltt}
\cvsID
%\end{alltt}
\normalsize
%\lfoot[\daspfooter]{\daspfooter}
%\lhead[\daspheaderl]{\daspheaderl}
%\chead[\daspheaderc]{\daspheaderc}
%\rhead[\daspheaderr]{\daspheaderr}
%\renewcommand{\sectionmark}[1]{\markboth{#1}{#1}}

\pagebreak
a
\vspace{2cm}
\begin{center}
\Large
{\bf Change record\\ \vspace{1cm}}
\normalsize
\daspchangerecord
\end{center}
\vspace{2cm}

\begin{center}
\Large
{\bf Notification list\\ \vspace{1cm}}
\normalsize
\daspnotificationlist
\end{center}

\pagebreak

\begin{center}
\Large
{\bf Acronyms and abbreviations\\ \vspace{1cm}}
\normalsize
\daspabbreviations
\end{center}

\pagebreak

\begin{center}
\Large
{\bf Applicable documents\\ \vspace{1cm}}
\normalsize

\daspapplicabledocs
\end{center}
\vspace{2cm}

\begin{center}
\Large
{\bf Reference documents \\ \vspace{1cm}}
\normalsize

\dasprefdocs
\end{center}

\pagebreak
\tableofcontents
\pagebreak


\section{Introduction}
This document is intended to provide details about the format of the
XML file used for specifying a simulation schema, and the method used
to convert it into a Python executable program, and run it.

\section{XML definition}
A module tag has the following attributes:
\begin{description}
\item[name]  Must be unique.
\item[parents] A dictionary of parent modules (names).
\item[providesFeedback] Set for reconstructor modules, default to zero.
\item[node] Two numbers separated by a period, e.g. 6.1  The first of
  these numbers represents the cray node (1-6 valid), the second
  represents the CPU number (starting at 1 and taken modulus the
  number of CPUs).  Processes on the same node
  can communicate via SHM.  A process on 6.1 will be run as a separate
  process from one on 6.3, but will be running on the same CPU.
  However, this is not recommended, unless you can be sure there will
  be speed benefits.
\item[object] The python constructor for this object,
  e.g. science.wfs.wfs.
\item[idstr] An identification string for this object (used for the
  parameter file).
\end{description}

Once the schema file has been prepared (probably using the simSetup
GUI), the simulation can be executed using \texttt{execsim schema.xml}
where schema.xml is the name of the schema file.  The execsim command
will parse this file, and create a Python program which forms the
basis of the simulation.  It will also determine the nodes on which
this simulation is to be run, and call the \texttt{mpirun} command.
Command line arguments that can be passed to \texttt{execsim} include
\texttt{generateOnly} (which will generate the Python code but not run
the simulation).  Any other command line arguments to given to
\texttt{execsim} will be passed straight to the simulation (for
example, --iterations=20).  The simulation will then be running.

When parsing the schema file, the \texttt{execsim} command will aim to
generate Python code which will have the following effects:
\begin{itemize}
\item Any remote communication send commands will happen as early as
  possible (to avoid holding up other processes).
\item Any remote communication get commands will happen as late as
  possible (to avoid waiting for the remote processes).
\item All objects with the same node (CPU) identifier will be placed
  in a single process and executed in turn.
\end{itemize}

\pagebreak
%if uncomment this, also change the makefile...
%\bibliography{references}

\printindex
\end{document}
