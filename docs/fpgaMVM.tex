\newcommand{\cvsID}{$ $Id: fpgaMVM.tex,v 1.11 2006/08/08 07:01:37 ali Exp $ (CVS)$}
\documentclass{article}
\usepackage{tabularx}
\newcommand{\mod}[1]{\texttt{#1}}
\usepackage{natbib}
\usepackage{graphicx}
\bibliographystyle{plainnat}
\usepackage{fancyheadings}
\usepackage{tabularx}
\usepackage{alltt, parskip, boxedminipage}
\usepackage{makeidx, multirow, longtable, tocbibind, amssymb}
%\usepackage{fullpage}
\makeindex
\usepackage[usenames]{color}
\definecolor{darkblue}{rgb}{0,0.05,0.35}

\usepackage[dvips, pagebackref, pdftitle={}, pdfcreator={epydoc 2.1}, bookmarks=true, bookmarksopen=false, pdfpagemode=UseOutlines, colorlinks=true, linkcolor=black, anchorcolor=black, citecolor=black, filecolor=black, menucolor=black, pagecolor=black, urlcolor=darkblue]{hyperref}
\setlength{\textheight}{21.5cm}
\setlength{\textwidth}{18cm}
\setlength{\hoffset}{-3.0cm}
\setlength{\footskip}{1.5cm}
\setlength{\headsep}{0.5cm}
\setlength{\voffset}{-2.5cm}
\newlength{\BCL} % base class length, for base trees.

\usepackage{everyshi}
 \makeatletter
 \let\totalpages\relax
 \newcounter{mypage}
 \EveryShipout{\stepcounter{mypage}}
 \AtEndDocument{\clearpage
    \immediate\write\@auxout{%
     \string\gdef\string\totalpages{\themypage}}}
 \makeatother

\newcommand{\daspfooter}{
\includegraphics[height=1cm]{durlogo.eps}
{\bf \large Centre for \AA dvanced Instrumentation}
}
\newcommand{\daspheaderl}{
\includegraphics[height=1cm]{durlogo.eps}
\begin{tabularx}{9cm}{c}
\Large {\bf \dasptitle} \\ \large {\bf(\daspdoctype)}\\
\rightmark\hspace{0.1cm}
\end{tabularx}
\vfill
}
\newcommand{\daspheaderc}{
}
\newcommand{\daspheaderr}{
\begin{tabular}{|l|l|}\hline
Doc. number: & \daspdocno \\ \hline
Release date: & \daspreleasedate \\ \hline
Issue number: & \daspissue \\ \hline
Page number: & Page \thepage \ of \totalpages \\ \hline
Author(s) & \daspauthorname \\ \hline
\end{tabular}


}

\pagestyle{fancy}
\cfoot[]{}
\lfoot[\daspfooter]{\daspfooter}
\lhead[\daspheaderl]{\daspheaderl}
\chead[\daspheaderc]{\daspheaderc}
\rhead[\daspheaderr]{\daspheaderr}
\renewcommand{\sectionmark}[1]{\markboth{#1}{#1}}

\newenvironment{Ventry}[1]%
  {\begin{list}{}{%
    \renewcommand{\makelabel}[1]{\texttt{##1:}\hfil}%
    \settowidth{\labelwidth}{\texttt{#1:}}%
    \setlength{\leftmargin}{\labelsep}%
    \addtolength{\leftmargin}{\labelwidth}}}%
  {\end{list}}


\begin{document}
\newcommand{\daspproject}{AO Simulation Project}
\newcommand{\dasptitle}{FPGA acceleration for MVM}
\newcommand{\daspdocno}{AOSIM-MVM-UoD-001}
\newcommand{\daspdoctype}{Internal}
\newcommand{\daspissue}{0.1.1}
\newcommand{\daspreleasedate}{\today}
\newcommand{\daspauthorname}{Alastair Basden}
\newcommand{\daspauthortype}{AO sim team member}
\newcommand{\daspapprovername}{Alastair Basden}
\newcommand{\daspapprovertype}{AO sim team member}
\newcommand{\daspreleasername}{Alastair Basden}
\newcommand{\daspreleasertype}{AO sim team member}
\newcommand{\daspreviewername}{Alastair Basden}
\newcommand{\daspreviewertype}{AO sim team member}
\newcommand{\daspchangerecord}{
\begin{tabular}{|l|l|l|l|}
\hline
Issue number & Release date & section(s) affected & Description of
change/remarks\\ \hline
0.1.0 & 051111 & All & First draft \\ \hline
0.1.1 & 051111 & All & Corrections made to first draft\\ \hline
\end{tabular}
}
\newcommand{\daspnotificationlist}{
Alastair Basden\\
Francois Assemat\\
Richard Wilson\\
Tim Morris\\
Tim Butterley\\
Ali Bharmal\\
}
\newcommand{\daspabbreviations}{
\begin{tabular}{rl}
AO & Adaptive Optics\\
ESO & European Southern Observatory\\
\end{tabular}
}
\newcommand{\daspapplicabledocs}{
\begin{tabular}{|l|l|l|}\hline
AD Number & Document title & Doc number/publication/location \\ \hline
AD01 & Simulation API document & AOSIM-API-UoD-001\\ \hline
AD02 & Simulation for dummies document & AOSIM-DUM-UoD-001\\ \hline
\end{tabular}
}
\newcommand{\dasprefdocs}{
\begin{tabular}{|l|l|l|}\hline
RD number & Document title & Doc number/publication/location \\ \hline
RD01 & Durham FPGA website & www.durham.ac.uk/rtcs.project \\ \hline
\end{tabular}
}
\title{\dasptitle}



\thispagestyle{empty}
%This next command provides the CVS tag.  If you want a cvs tag on
%your document, add the following line at the start of the document,
%after replacing the & signs with dollar signs...
%\newcommand{\cvsID}{& &Id& (CVS)&}
\providecommand{\cvsID}{CVS ID not provided: document made on \today}

\begin{center}
\includegraphics{durlogo.eps}
\end{center}
\vspace{0.5cm}
\Huge
\begin{center}
\daspproject\\
\end{center}
\Large
\vspace{1cm}


{\bf 
\begin{tabular}{ll}
Document title: & \dasptitle \vspace{0.5cm}\\ 

Documentation number: & \daspdocno \vspace{0.5cm}\\ 

Document type: & \daspdoctype \vspace{0.5cm}\\ 

Issue number:& \daspissue \vspace{0.5cm}\\ 

Release date: & \daspreleasedate \\ 

\end{tabular}
}

\normalsize
\vfill

\begin{tabular}{|l|l|l|p{5cm}|}
\hline
Document & \daspauthorname & Signature &\\
prepared by & \daspauthortype & and date &\\ \hline
Document & \daspapprovername & Signature &\\
approved by & \daspapprovertype & and date &\\ \hline
Document & \daspreleasername & Signature &\\
released by & \daspreleasertype & and date &\\ \hline
Document & \daspreviewername & Signature &\\
reviewed by & \daspreviewertype & and date &\\ \hline
\end{tabular}

\small
%\begin{alltt}
\cvsID
%\end{alltt}
\normalsize
%\lfoot[\daspfooter]{\daspfooter}
%\lhead[\daspheaderl]{\daspheaderl}
%\chead[\daspheaderc]{\daspheaderc}
%\rhead[\daspheaderr]{\daspheaderr}
%\renewcommand{\sectionmark}[1]{\markboth{#1}{#1}}

\pagebreak
a
\vspace{2cm}
\begin{center}
\Large
{\bf Change record\\ \vspace{1cm}}
\normalsize
\daspchangerecord
\end{center}
\vspace{2cm}

\begin{center}
\Large
{\bf Notification list\\ \vspace{1cm}}
\normalsize
\daspnotificationlist
\end{center}

\pagebreak

\begin{center}
\Large
{\bf Acronyms and abbreviations\\ \vspace{1cm}}
\normalsize
\daspabbreviations
\end{center}

\pagebreak

\begin{center}
\Large
{\bf Applicable documents\\ \vspace{1cm}}
\normalsize

\daspapplicabledocs
\end{center}
\vspace{2cm}

\begin{center}
\Large
{\bf Reference documents \\ \vspace{1cm}}
\normalsize

\dasprefdocs
\end{center}

\pagebreak
\tableofcontents
\pagebreak

\section{Scope}
This document describes the details of the decision not to implement
matrix-vector multiplication algorithms in the Cray XD1 FPGA
acceleration available with the AO simulation software package which
is funded by the rolling grant.

This document is relevant to anyone who is involved with the
development of the AO simulation software.  For a brief description on
how to set up your own AO simulation, please read \citet{dummies}.
This document can also be used by FPGA programmers who which to
consider the various decisions made.

\subsection{Overview}
A matrix vector multiplication operation can be performed in a highly
parallelised way, and so would appear ideal for implementing in an
FPGA.  Additionally, during wavefront reconstruction, the main place
where a matrix-vector multiplication (MVM) may be used in an AO
simulation, the matrix is essentially constant, giving us the
opertunity to achieve the best possible performance since the matrix
can be passed to the FPGA only once.  

However, even with all these advantages, it has been decided that it
is not worth implementing the MVM algorithm and the reasons for this
are provided here.

\section{MVM performance}
Given that the matrix in a wavefront reconstruction MVM does not
change, it can be stored either in internal FPGA memory, in external
memory local to the FPGA or in the host processor memory.  The
Virtex-2 Pro in the Cray XD1 contains 464~KB internal memory and 16~MB
external local memory.  We can therefore separate the MVM problem into
three cases:

\begin{enumerate}
\item The matrix is less than 464~KB and can be stored in internal
memory.
\item The matrix is less than 16~MB and can be stored in local
external memory.
\item The matrix is greater than 16~MB and has to be stored in host
processor memory.
\end{enumerate}

We assume that 32 bit floating point values will be used for the
reconstruction (best case for FPGA performance improvement) and that
the vector has a size $N_v$, while the matrix has dimensions $(N_y,
N_v)$.  If a sparse matrix implementation is to be used, this will not
affect the timing results considered here, only the algorithm
implementation.  If $S$ is the fraction of the matrix used (the
sparsity factor), then the matrix can be compressed to dimensions
$(N_y, N_v\times S)$.  It is the size of this matrix (sparse or
otherwise) that is considered here.

\subsection{Processor timings}
Table~\ref{tab:cputime} gives the timings for MVM for a few different
(important) matrix and vector sizes.
\begin{table}
\begin{tabular}{llll}\hline
Vector dimension & CPU processing time / s & Comment & FPGA processing
time\\ \hline
344 & 0.000322 & FPGA internal memory & 25-55 $\mu$s\\
2048 & 0.00968 & FPGA local external memory & 2.6 ms\\
16384 & 0.627 & Host processor memory & 0.67 s\\ \hline
\end{tabular}
\caption{A table showing CPU and FPGA processing timess for MVM, assuming a
square matrix.  The FPGA timings are optimistic, the CPU timings are actual.}
\label{tab:cputime}
\end{table}

\subsection{Matrix in internal memory}
The Virtex-2 Pro-50 has 232 internal block memory banks (each 2~KB).  
If the matrix is stored in internal memory, it is therefore possible
to access 232 new matrix values every clock cycle (assuming no block
memory is required for anything else, probably a poor assumption, but
best case).  It the matrix is square, a $344\times344$ matrix is the
largest that can be stored.  It will therefore take two clock cycles
to read each matrix column or row.  The vector will be passed from
host processor memory into the FPGA sequentially (the maximum
bandwidth will allow two 32-bit floating point numbers every clock
cycle at 199~MHz).  As each new vector value arrives, it can be
multiplied by all of the matrix elements, and the intermediate results
stored.  It will take two clock cycles to read all of the necessary
matrix elements, so taking $344\times2+l=688+l$ clock cycles to perform
the entire MVM, where $l$ is the pipeline latency of the floating
point multiplication and addition operations, typically of order 10
clock cycles (so can be ignored here).  This will therefore take
approximately 3.5~$\mu$s to perform.  From the wavefront sensing
pipeline, we know there is a host processor memory access latency of
about 20~$\mu$s, so the total computation time is expected to be about
23~$\mu$s, about 14 times faster than the software multiplication given
in table~\ref{tab:cputime}.

It should be noted that this assumes that 232 floating point
multiplication and addition cores can be implemented simultaneously in
the FPGA.  This is unlikely (maybe a factor of 10 too large), and so
realistically, the hardware will take a factor of 10 times longer,
maybe 55~$\mu$s including the latency, giving a performance
improvement of only about 6 times.   

If the matrix can be stored internally as 16 (or 18) bit floating
point, the performance can be further improved, maybe by up to a
factor of two, though this depends on the number of floating point
arithmetic cores that can be implemented simultaneously.

\subsection{Matrix in local external memory}
The Cray XD1 FPGAs have four external memory banks, each 64 bits wide,
meaning that 8 32-bit floating point numbers can be accessed each
clock cycle.  A matrix of size $2048\times2048$ is the largest that
can be stored in the external memory.  We must therefore wait
$2048/8=256$ clock cycles to read all elements in a given column of
the matrix, so taking 256 clock cycles (plus a pipelined latency) to
operate on each new vector input value.  The total computation time
will therefore be (ignoring the initial pipeline latency)
$256\times2048=524288$ clock cycles, or 0.0026 seconds.  This is a
factor of about four times faster than the software multiplication
time of 0.00968 seconds.

If the matrix is stored as 16 (or 18) bit floating point numbers, then
16 such numbers can be accessed each clock cycle.  We must then wait
for 128 clock cycles to read all elements of a given column of a
$2048\times2048$ matrix, takling 262144 clock cycles (0.0013
seconds).  This is therefore a factor of about eight times faster than
the software multiplication time.

\subsection{Matrix in host processor memory}
If the matrix has to be stored in host processor memory, the MVM will
be implemented by first loading the vector into the FPGA internal
memory.  The matrix will then be loaded into the FPGA to be operated
on at two floating point values per clock cycle, taking a total of
about $16384\times16384/2$ clock cycles, or 0.67 seconds for the MVM
to complete.  We have chosen a matrix here to be 1~GB in size, though
in practice, they could be larger, though this would have little
affect on the performance differences.  It should be noted that a MVM
of this size will be performed faster on a CPU, and that the hardware
FPGA timings are optimistic (in practice, the memory transfer to the
FPGA from the processor memory is only about 85\% of the theoretical
maximum assumed here).  

If the matrix can be stored as 16 bit floating point numbers then four
can be loaded into the FPGA every clock cycle, taking 0.34 seconds for
the MVM to complete for a $16384\times 16384$ matrix.  This is then
slightly better than the CPU computation time.  

\section{Design considerations}
We have seen that when the matrix in a MVM is small, some performance
gain can be achieved by using an FPGA instead of a CPU.  However, once
the matrix becomes too large to store in local FPGA memory, the CPU is
able to perform the MVM in less time.  

\subsection{ELT instrument considerations}
A planet finder instrument for an ELT may have $500\times500$
subapertures.  The (non-sparse) reconstruction matrix will then
typicall have dimensions of $(500\times500\times2,
500\times500\times2)$ assuming there are as many actuators as
subapertures.  This would take 1~TB of memory to store.  If this
matrix could be assumed to be sparse, with a sparsity factor of 0.001,
this would still take 1~GB to store, and so the MVM would perform
faster in software.  

A more standard classical AO instrument on an ELT may have
$75\times75$ subapertures, requiring 500~MB data for storage.  A
typical sparsity factor might be somewhere between 0.02 -- 0.05,
meaning that this matrix may or may not be storable in the FPGA
external local memory.  At best, a performance gain of 4 times could
be achieved by using the FPGAb (or 8 times, if the matrix could be
stored as 16 bit floating point).  An MCAO or other more complex system
would require a few times more memory than this, and so that matrix
would have to be stored in host processor memory, giving no
performance gain when using the FPGA.

\subsection{VLT instrument considerations}
Next generation instruments for VLT systems may have up to
$30\times30$ subapertures (for example, HOT, though this is not
sky-fieldable).  The reconstruction matrix for such a system would be
about 13~MB in size, storable in the FPGA local external memory,
giving a performance improvement of a factor of about four over a CPU
implementation.  If this matrix was assumed sparse (S=0.03), it
could be stored in FPGA internal memory, giving an even large
performance improvement.  

\subsection{Matrix precision}
The precision of the reconstruction matrix is crucial if the FPGA is
to be used to give preformance improvements.  If each sub-aperture
contains $8\times8$ pixels, then a 10 bit centroid value would be
sufficient to locate the centroid location to within less than 0.01 of
a pixel.  There is therefore little point in having a reconstructor
matrix with more precision, though floating point should be used to
conserve dynamic range.  A 10 bit mantissa, and 6 bit exponent may
therefore be sufficient to store this matrix.  If this is to be stored
internally in the FPGA, a further 2 bits are available since these can
be stored in the memory parity areas.

\section{Conclusions}
Implementation of a MVM algorithm in an FPGA would require some work.
Additionally, the differences between an algorithm using internal and
external memory would be sufficient to require a significant amount of
work.  Because of the unsuitability of the FPGA algorithm for the
ELT cases, it is probably not worth spending time to implement in
hardware.  Even for VLT cases, the performance improvement is only a
factor of four-six, small for a lot of effort.  It is therefore
probably not worth implementing an FPGA based MVM algorithm for the
XD1.  We are aiming for an ELT scale simulator, so the ELT
reconstruction is the important consideration.

If you disagree with any of these assumptions, calcuations of
conclusions, do please discuss it.

\pagebreak
\bibliography{references}


\printindex
\end{document}
