\documentclass{article}
\usepackage{natbib}
\usepackage{graphicx}
\bibliographystyle{plainnat}
\usepackage{fancyheadings}
\usepackage{tabularx}
\usepackage{alltt, parskip, boxedminipage}
\usepackage{makeidx, multirow, longtable, tocbibind, amssymb}
%\usepackage{fullpage}
\makeindex
\usepackage[usenames]{color}
\definecolor{darkblue}{rgb}{0,0.05,0.35}

\usepackage[dvips, pagebackref, pdftitle={}, pdfcreator={epydoc 2.1}, bookmarks=true, bookmarksopen=false, pdfpagemode=UseOutlines, colorlinks=true, linkcolor=black, anchorcolor=black, citecolor=black, filecolor=black, menucolor=black, pagecolor=black, urlcolor=darkblue]{hyperref}
\setlength{\textheight}{21.5cm}
\setlength{\textwidth}{18cm}
\setlength{\hoffset}{-3.0cm}
\setlength{\footskip}{1.5cm}
\setlength{\headsep}{0.5cm}
\setlength{\voffset}{-2.5cm}
\newlength{\BCL} % base class length, for base trees.

\usepackage{everyshi}
 \makeatletter
 \let\totalpages\relax
 \newcounter{mypage}
 \EveryShipout{\stepcounter{mypage}}
 \AtEndDocument{\clearpage
    \immediate\write\@auxout{%
     \string\gdef\string\totalpages{\themypage}}}
 \makeatother

\newcommand{\daspfooter}{
\includegraphics[height=1cm]{durlogo.eps}
{\bf \large Centre for \AA dvanced Instrumentation}
}
\newcommand{\daspheaderl}{
\includegraphics[height=1cm]{durlogo.eps}
\begin{tabularx}{9cm}{c}
\Large {\bf \dasptitle} \\ \large {\bf(\daspdoctype)}\\
\rightmark\hspace{0.1cm}
\end{tabularx}
\vfill
}
\newcommand{\daspheaderc}{
}
\newcommand{\daspheaderr}{
\begin{tabular}{|l|l|}\hline
Doc. number: & \daspdocno \\ \hline
Release date: & \daspreleasedate \\ \hline
Issue number: & \daspissue \\ \hline
Page number: & Page \thepage \ of \totalpages \\ \hline
Author(s) & \daspauthorname \\ \hline
\end{tabular}


}

\pagestyle{fancy}
\cfoot[]{}
\lfoot[\daspfooter]{\daspfooter}
\lhead[\daspheaderl]{\daspheaderl}
\chead[\daspheaderc]{\daspheaderc}
\rhead[\daspheaderr]{\daspheaderr}
\renewcommand{\sectionmark}[1]{\markboth{#1}{#1}}

\newenvironment{Ventry}[1]%
  {\begin{list}{}{%
    \renewcommand{\makelabel}[1]{\texttt{##1:}\hfil}%
    \settowidth{\labelwidth}{\texttt{#1:}}%
    \setlength{\leftmargin}{\labelsep}%
    \addtolength{\leftmargin}{\labelwidth}}}%
  {\end{list}}

\newcommand{\mod}[1]{\texttt{#1}}
\begin{document}
%\include{introductiontitle}
\newcommand{\daspproject}{Durham AO Simulation Platform}
\newcommand{\dasptitle}{Introduction to DASP}
\newcommand{\daspdocno}{DASP-INT-UoD-001}
\newcommand{\daspdoctype}{Internal}
\newcommand{\daspissue}{0.1.1}
\newcommand{\daspreleasedate}{\today}
\newcommand{\daspauthorname}{Alastair Basden}
\newcommand{\daspauthortype}{AO sim team member}
\newcommand{\daspapprovername}{Alastair Basden}
\newcommand{\daspapprovertype}{AO sim team member}
\newcommand{\daspreleasername}{Alastair Basden}
\newcommand{\daspreleasertype}{AO sim team member}
\newcommand{\daspreviewername}{Alastair Basden}
\newcommand{\daspreviewertype}{AO sim team member}
\newcommand{\daspchangerecord}{
\begin{tabular}{|l|l|l|l|}
\hline
Issue number & Release date & section(s) affected & Description of
change/remarks\\ \hline
0.1.0 & 160209 & All & First draft \\ \hline
\end{tabular}
}
\newcommand{\daspnotificationlist}{
Alastair Basden\\
}
\newcommand{\daspabbreviations}{
\begin{tabular}{rl}
AO & Adaptive Optics\\
\end{tabular}
}
\newcommand{\daspapplicabledocs}{
\begin{tabular}{|l|l|l|}\hline
AD Number & Document title & Doc number/publication/location \\ \hline
\end{tabular}
}
\newcommand{\dasprefdocs}{
\begin{tabular}{|l|l|l|}\hline
RD number & Document title & Doc number/publication/location \\ \hline
\end{tabular}
}
\title{\dasptitle}



\thispagestyle{empty}
%This next command provides the CVS tag.  If you want a cvs tag on
%your document, add the following line at the start of the document,
%after replacing the & signs with dollar signs...
%\newcommand{\cvsID}{& &Id& (CVS)&}
\providecommand{\cvsID}{CVS ID not provided: document made on \today}

\begin{center}
\includegraphics{durlogo.eps}
\end{center}
\vspace{0.5cm}
\Huge
\begin{center}
\daspproject\\
\end{center}
\Large
\vspace{1cm}


{\bf 
\begin{tabular}{ll}
Document title: & \dasptitle \vspace{0.5cm}\\ 

Documentation number: & \daspdocno \vspace{0.5cm}\\ 

Document type: & \daspdoctype \vspace{0.5cm}\\ 

Issue number:& \daspissue \vspace{0.5cm}\\ 

Release date: & \daspreleasedate \\ 

\end{tabular}
}

\normalsize
\vfill

\begin{tabular}{|l|l|l|p{5cm}|}
\hline
Document & \daspauthorname & Signature &\\
prepared by & \daspauthortype & and date &\\ \hline
Document & \daspapprovername & Signature &\\
approved by & \daspapprovertype & and date &\\ \hline
Document & \daspreleasername & Signature &\\
released by & \daspreleasertype & and date &\\ \hline
Document & \daspreviewername & Signature &\\
reviewed by & \daspreviewertype & and date &\\ \hline
\end{tabular}

\small
%\begin{alltt}
\cvsID
%\end{alltt}
\normalsize
%\lfoot[\daspfooter]{\daspfooter}
%\lhead[\daspheaderl]{\daspheaderl}
%\chead[\daspheaderc]{\daspheaderc}
%\rhead[\daspheaderr]{\daspheaderr}
%\renewcommand{\sectionmark}[1]{\markboth{#1}{#1}}

\pagebreak
a
\vspace{2cm}
\begin{center}
\Large
{\bf Change record\\ \vspace{1cm}}
\normalsize
\daspchangerecord
\end{center}
\vspace{2cm}

\begin{center}
\Large
{\bf Notification list\\ \vspace{1cm}}
\normalsize
\daspnotificationlist
\end{center}

\pagebreak

\begin{center}
\Large
{\bf Acronyms and abbreviations\\ \vspace{1cm}}
\normalsize
\daspabbreviations
\end{center}

\pagebreak

\begin{center}
\Large
{\bf Applicable documents\\ \vspace{1cm}}
\normalsize

\daspapplicabledocs
\end{center}
\vspace{2cm}

\begin{center}
\Large
{\bf Reference documents \\ \vspace{1cm}}
\normalsize

\dasprefdocs
\end{center}

\pagebreak
\tableofcontents
\pagebreak


\section{Introduction}
This guide is intended as an introduction to DASP, and will aid you in
creating and using AO simulations.  It includes information about how
to configure simulations along with a description of necessary
parameter names.

\section{Installation}
See the INSTALL file.  Basically, install dependencies, type
``make'', then edit your environment variables as instructed.

\section{Creating a first simulation}
Run \begin{verbatim}daspbuilder.py\end{verbatim}.  This will lead you
  through the process of creating a simulation (SCAO, GLAO, MCAO,
  LTAO, MOAO).  Once created, follow the instructions to run this
  simulation.  To investigate the simulation,
  use \begin{verbatim}daspsetup.py XXX.xml\end{verbatim} where XXX.xml
  is the name of the XML file created.  Also, read the params.py file
  created, to see the current configuration of the simulation.

\section{Running a DASP simulation}
Usually, this is a case of executing:
\begin{verbatim}
python SIM.py params.py
\end{verbatim}
where SIM.py is the name of your simulation file.  There are various
command line parameters that can be seen by appending --help to the
command.  The most useful of these is --iterations=X, to specify the
number of iterations to run for.

\section{Overview}
A DASP simulation is made from a collection of different simulation
modules, e.g.\ to model the atmosphere, to model DMs, to model
wavefront sensors, science PSF instruments, and wavefront
reconstruction models.  There are many such modules, and it is simple
to create your own.  These are found in the science/ directory.  The most useful are:
\begin{itemize}
\item iscrn: Generates one or more infinite length atmospheric
  turbulence phase screens, which are then translated across the
  telescope pupil.
\item iatmos: Takes these phase screens and projects along a given
  line of sight, to generate the phase aberrations at the telescope
  pupil in a given direction.  Each iatmos module therefore produces
  phase for 1 line of sight, for 1 wavelength.
\item xinterp\_dm: A DM module which is based on cubic spline
  interpolation of actuators to produce a continuous phase sheet DM
  (by default - other surface models are also available).  
\item zdm: A Zernike based DM.  Using the first 3 modes will give you
  a tip-tilt mirror (piston is ignored).
\item wfscent: Produces wavefront sensor images from the corrected or
  uncorrected phase, adds noise, and then computes slope measurements.
\item tomoRecon: A tomographic reconstructor module, which will also
  do non-tomographic reconstruction.
\item science: A module for generation of science PSF images, to
  obtain simulation performance
\end{itemize}

\section{Parameters}
Here we discuss parameters used by the simulation.
\subsection{Simulation parameters}
\subsubsection{tstep}
The base time step of the simulation, in seconds.  Usually specified
as $\frac{1}{framerate}$.
\subsubsection{AOExpTime}
The simulation exposure time.  This, along with tstep, define the
number of iteration that the simulation will be run for.  If set to
zero, the simulation will run indefinitely, until asked to stop by the
user.  

\subsection{Telescope parameters}
\subsubsection{telDiam}
Telescope diameter in m.
\subsubsection{telSec}
Central obscuration diameter in m.
\subsubsection{npup}
Number of phase elements across the pupil.
\subsubsection{pupil}
The telescope pupil function.  A 2D bitmap defining which areas of the
phase fall within the telescope pupil.  The telescope pupil can be
different for different simulation modules if necessary (for example,
a science module may not want to see the edges of a hexagonal based
primary mirror).  For convenience, this can be created using the
util.tel.Pupil object, e.g.:
\begin{verbatim}
import util.tel
pupil=util.tel.Pupil(npup,npup/2,npup/2.*telSec/telDiam)
\end{verbatim}
Spiders etc.\ can also be added.

\subsection{Atmospheric parameters}
\subsubsection{r0}
Global integrated Fried's parameter, in m.  
\subsubsection{l0}
Outer scale, in m.
\subsubsection{layerDict}
A dictionary of atmospheric layers.  The keys will be a text string
that you use to identify the layers.  Typically, something like L0,
L1, L2, etc.

The values will be {\texttt layer} objects (from the util.atmos
module) used to define the characteristics of each layer.  

The layer objects take parameters:
\begin{itemize}
\item height: The height of the layer in m above the telescope.
\item direction: The direction in which the layer translates, in
  degrees.
\item speed: The speed of the layer, in ms$^{-1}$.
\item strength: The relative strength of the layer, used to compute
  $r_0$ for this layer.  Strengths of all layers should add up to 1.
\item seed: Optional, a seed for the random number generator if you
  wish to have repeatable simulations.
\end{itemize}

Example:
\begin{verbatim}
layerDict={"L0":layer(height=0.,direction=0.,speed=7.5,strength=0.7),
           "L1":layer(height=4000.,direction=330.,speed=12.5,strength=0.2),
          }
\end{verbatim}

\subsubsection{atmosGeom}
An atmosphere and geometry object which encompasses layer heights and
source directions (from the util.atmos module).  This takes the
layerDict parameter, values within the wfsOverview and sciOverview
objects, the npup value, telDiam, r0 and l0.  

Example:
\begin{verbatim}
atmosGeom=geom(layerDict,wfsOverview.values()+sciOverview.values(),npup,npup,telDiam,r0,l0)
\end{verbatim}


\subsection{WFS parameters}
WFS parameters are encapsulated in a NGS or LGS object (from the
util.guideStar module).  A number of these objects are then used to
create a wfsOverview object, giving an overview of all WFS objects
within the simulation.

\subsubsection{wfsOverview}
The wfsOverview object (util.guideStar) takes a Python dictionary with
keys corresponding to the name of the WFS module ID string (typically
a label for the direction of interest) and values being a LGS or NGS
object.

The LGS object takes the following parameters (only the most common
ones are given here - to see the rest, look at util.guideStar.py or in
Python do help(util.guideStar.LGS)).

\begin{itemize}
\item idstr: The identification string for this WFS (see the
  daspsetup.py input file).
\item nsubx: The number of sub-apertures in one direction (the WFSs
  are assumed to be square).
\item theta: The offaxis direction within the field of view at which
  the WFS is pointed, in arcseconds.  (i.e.\ r in polar coordinates)
\item phi: The angle in degrees (i.e.\ $\theta$ in polar coordinates)
  specifying the position of the guide star within the telescope field
  of view.  An on-axis target would have theta=0, phi=0.  A target 10
  arcseconds off-axis would have theta=10, and phi set depending where
  on a 10 arcsecond radius circle the star was.
\item height: The height of the LGS (e.g. typically 90~km for a sodium
  laser).
\item phasesize: The number of phase pixels per sub-aperture.
\item pupil: The pupil function of the WFS.
\item minarea: Optional.  The minimum fraction of sub-aperture area to be light
  sensitive for the sub-aperture to be counted as usable. This
  defaults to 0.5 (50\%), i.e. 50\% of the sub-aperture must not be
  vignetted for it to be used.  
\item sig: Optional.  The flux, in detected photons per sub-aperture per frame.
\item launchDist: Optional.  Distance from the centre of the pupil to the laser
  launch location, in m.  This will be zero for centre-launched
  lasers (along with launchTheta).
\item launchTheta: Optional.  Angle, in degrees, specifying the laser launch
  location (along with launchDist).
\item sourcelam: Optional.  Wavelength, in nm, of the WFS.
\item phslam: Optional.  Wavelength of incident phase, in nm.  If not specified,
  defaults to the WFS wavelength.  
\item reconList: Optional.  List of reconstructor object names that this WFS
  provides reconstruction for.  Typically will just have one entry,
  ``recon''.
\item nimg: Optional.  Number of CCD pixels (in each direction) for
  each sub-aperture.
\item nfft: Optional.  Size of the array to be used during computation
  of the FFT.  Typically at least twice phasesize.
\item readoutNoise: Optional.  Readout noise in electrons RMS.
\item lgsPsf: Optional.  A PSF for each sub-aperture to specify LGS
  elongation.
\item spotpsf: Optional.  A single PSF to be replicated for each
  sub-aperture.  
\item subapFlag: Optional.  An array specifying which sub-apertures
  are used.  If not specified, this is calculated automatically from
  the vignetting.
\item integSteps: Optional, default 1.  The number of time steps over
  which WFS integration should be performed.  
\item skyBrightness: Optional, a value or array specifying the sky
  brightness, i.e.\ the background image.
\item bglevel: Optional, a value or array specifying the detector
  background level (i.e.\ bias level).
\end{itemize}

The NGS object is similar, but does not specify a height (assumed to
be at infinity).  The lgsPsf, launchDist and launchTheta parameters
are also not available.  

Example, for a on-axis LGS and a low order off-axis NGS:
\begin{verbatim}
wfsOverview=util.guideStar.wfsOverview({
    "a":util.guideStar.LGS("a",nsubx=10,theta=0.,phi=0.,height=90000.,phasesize=8,pupil=pupil),
    "b":util.guideStar.NGS("b",nsubx=2,theta=10.,phi=45.,phasesize=40,pupil=pupil)
    })
\end{verbatim}

\subsection{Science parameters}
The science (PSF generation and performance analysis) parameters are
encapsulated within a sciOverview object.

\subsubsection{sciOverview}
The sciOverview object (in the util.sci module) takes a dictionary of
sciInfo objects (also from the util.sci module).  These objects take
the following parameters (most are given here, for others look at
sci.py, or in Python, do help(util.sci.sciInfo)):

\begin{itemize}
\item idstr: The ID string for this science object (see the
  daspsetup.py input file).
\item theta, phi: Polar coordinates of the position within the field
  of view of this science object.  theta is in arcseconds, phi in
  degrees.
\item pupil: Telescope pupil function for this science object.
\item sourcelam: Wavelength of this science object.
\item phslam: Optional.  Wavelength of the phase input for this
  science object.  Defaults to the sourcelam value.
\item psfFilename: Optional.  Filename for output PSFs.  If not
  specified, will not be saved.  This should be a FITS file, which
  gets appended to for consecutive science objects and simulation runs.
\item summaryFilename: Optional.  Filename for results (Strehl, etc).
  Defaults to results.csv.  This will be appended to for consecutive
  science objects and simulation runs.
\item calcRMS: Optional.  Whether to calculate the RMS of input
  phase.  Defaults to zero.
\end{itemize}

Example, for an on-axis and off-axis target:
\begin{verbatim}
sciOverview=util.sci.sciOverview({
    "a":util.sci.sciInfo("a",theta=0.,phi=0.,pupil=pupil,sourcelam=1650.),
    "b":util.sci.sciInfo("b",theta=10.,phi=45.,pupil=pupil,sourcelam=1650.),
})
\end{verbatim}

\subsection{DM configuration}
DM parameters are encapsulated in a dmOverview object.  

\subsubsection{dmOverview}
The dmOverview object (in the util.dm module) takes a list of dmInfo
objects (in the util.dm module), and also the atmosGeom parameter.

The dmInfo objects take the following parameters (the most important
ones are given here, for others look at dm.py, or in Python do
help(util.dm.dmInfo)):
\begin{itemize}
\item label: The DM label, as also given in the daspsetup.py input
  file.  This identifies the DM.
\item idlist: A list of directions that this DM will cover.  This is
  used in MCAO systems, and for virtual DMs to compute the size of the
  DM so that it is large enough to cover the requested field of view.
\item height: The DM height in m above the telescope.
\item nact: The number of actuators in 1 direction (or modes for modal
  mirrors).
\item minarea: Optional, default 0.2.  Vignetting of an actuator for it to be considered
  valid.
\item actuatorsFrom: Optional, default ``reconstructor''.  The name of
  the reconstructor object (see the daspsetup.py input file) which
  provides actuator commands for this DM (required since some
  simulations might have several reconstructor objects).
\item reconstructList: Optional, default ``all''.  A list of names of the reconstructor objects which are computing
actuators for this DM.  See the daspsetup.py input file.
\item zonalDM: Optional, default 1.  Set to 0 for a modal DM.
\end{itemize}

Example (for a 21x21 actuator DM and a TT mirror):
\begin{verbatim}
dmOverview=util.dm.dmOverview([
 dmInfo('dm',["a","b"],height=0.,nact=21,minarea=0.1,actuatorsFrom="darc",reconstructList="all")
 dmInfo('tt',["a","b"],height=0.,nact=3,actuatorsFrom="darc",reconstructList="all")
    ], atmosGeom)
\end{verbatim}

\subsection{Reconstructor configuration}
The standard reconstructor within DASP is tomoRecon.py.  This performs
a matrix-vector multiplication reconstruction with the contents of the
matrix optionally defined by the user.  However, this module is also
capable of taking an interaction matrix, and computing the
pseudo-inverse to give a least-squares control matrix.  Should another
control matrix be required (e.g.  minimum variance), this is usually
computed externally by the user, using the interaction matrix and
other information that they have about the system, e.g. WFS noise
statistics, turbulence statistics, etc.

Parameters that this module can take include:
\begin{itemize}
\item rcond: The conditioning value for the pseudo inverse (see
 the numpy.linalg.pinv method for further information).  This will
 typically range from 0.1 to $10^{-6}$ depending on system type and
 size.
\item recontype: The type of the reconstructor.  ``pinv'' is most
  commonly used, though other more advanced options are available.
\item pokeval: The size of the pokes used during interaction matrix
  computation.  This should probably be a value per DM, but that is
  not yet implemented (though the fix is fairly trivial).
\item gainFactor: The closed-loop gain.  Can be a single value, or a
  value per actuator.
\item decayFactor: An anti-windup factor to prevent unseen modes
  building up on the DMs, typically something just less than 1.  Can
  be a single value or a value per actuator.
\item computeControl: A flag, whether to automatically compute a
  control matrix after creating an interaction matrix.  For large AO
  systems, this should probably be 0, since numpy.linalg.pinv() is not
  efficient, and you would be better using a HPC library outside of
  DASP to perform this algorithm.
\item reconmxFilename: FITS filename for the control matrix (either
  that is to be created, or to be loaded).
\item pmxFilename: FITS filename for the interaction (poke) matrix.
\end{itemize}

\subsection{Other configurations}
Other simulation modules can require additional parameters.  These are
not documented here because future requirements of these modules may
change.  However, to identify the parameter requirements for a given
module, searching for the string ``getVal'' within the module will
reveal that parameters (including optional ones) which are required.

\subsection{Structure and separation of parameters}
By default, all parameters created in the parameter file (assuming the
Python version, not XML files) are global, and accessible to all
simulation modules.  However, a hierarchy exists to enable modules to
preferentially select parameters unique to them.  

To set parameters that are unique to all modules of one type
(e.g.\ tomoRecon), within the parameter file, one would do:
\begin{verbatim}
this.tomoRecon=new()
this.tomoRecon.gainFactor=0.5
this.tomoRecon.rcond=0.01
etc
\end{verbatim}
Therefore, if other modules tried to get the value of gainFactor, they
would fail (unless it was also specified globally).

To set parameters unique to a module with a specific identification
string (idstr as given in the daspsetup.py input file) you would use
(for an idstr of ``scao''):
\begin{verbatim}
this.tomoRecon_scao=new()
this.tomoRecon_scao.gainFactor=0.5
this.tomoRecon_scao.rcond=0.01
etc
\end{verbatim}

This allows a hierarchy to be built up.  For example:
\begin{verbatim}
myval=1
this.tomoRecon=new()
this.tomoRecon.myval=2
this.tomoRecon_scao=new()
this.tomoRecon_scao.myval=3
\end{verbatim}
This would allow a tomoRecon module with an idstr equal to ``scao'' to
get the value of ``myval'' as 3.  For other tomoRecon modules, the
value of ``myval'' would be 2.  For modules that are not tomoRecon
modules, the value of ``myval'' would be 1.  This approach allows
flexibility, and enables unique parameter values for different modules.



\section{Simulation design}
The daspbuilder.py tool is the recommended way to create a new
simulation.  However, should your requirements differ or you want
something more advanced (e.g.\ with different modules, different AO
designs, etc.), you can create a simulation in two steps.  First
create the simulation runtime file, typically using the {\tt
  daspsetup} GUI (or by hand).  Then create a corresponding parameter
file.

\subsection{Creating a simulation runtime file}
It is recommended that you use the \texttt{daspsetup} GUI for
simulation creation.  This has a separate manual.  

\section{Setting up the parameter file}
\label{sect:paramfile}
You now have a simulation file that can be run by the python
interpreter.  However, before doing this, you must create the
parameter file, which can be either in XML or Python format.  The XML
format is now depreciated, so we recommend that a Python format is used.

A good starting point for a parameter file is one created using
daspbuilder.py.  From here, you can then edit to suit your requirements.

\section{Running the simulation}
Once the parameter file is set up, you are ready to run the
simulation.  This can be done by executing the following command:

\begin{verbatim}
python mysimulation.py
or
aosim mysimulation.py params.py
\end{verbatim}

If you have a simulation that uses MPI (Message Passing Interface) to
distribute the simulation across several different computational nodes
(which can be selected within daspbuilder.py or in daspsetup.py), then
you need to launch the simulation in a slightly different way, as
given in the first line of the simulation file.  Depending on your
version of MPI libraries, you will need to use either mpipython
(Scientific module), or standard python (mpi4py module).  The first
line of your simulation file will give a suggested way to run with
MPI.  For example, to run a simulation on 4 nodes, you would use:
\begin{verbatim}
mpirun -np 4 python mysimulation.py params.py
\end{verbatim}
For further information, investigate the mpirun command, e.g.\ to
specify which compute nodes the simulation should run on.


\subsection{Command line arguments}
There are a number of command line arguments that can be passed to a
simulation.  These include
\begin{enumerate}
\item -h or --help to display a help message (including an up to date
  list of command line arguments).
\item --start-paused to place the simulation in an initial paused
  state.
\item --param-file=FILENAME to specify a parameter file name (the
  default is params.xml).
\item --iterations=N to specify the number of iterations the simulation
  is run for (the default is infinity if not specified in the
  parameter file).
\item --param=''valid python text string'' to change a parameter on the fly,
  for example, you could use --param=''this.globals.wfs\_n=16'' to
  change the value of wfs\_n to 16, but without having to change the
  parameter file.  Note, this should be used with care, because there
  will probably be no record of the change (unless the commandline
  switch is saved).  These changes are inserted after the parameter
  file has been executed, so any dependencies within it won't be
  changed.
\item --init=''name=value;name=value'' to specify name and value pairs
  to be inserted before parameter file execution.  If the parameter
  file is written correctly, these values will then be picked up and used.
\item --batchno=VALUE can be used to run a specific batch (specific
  parameters from the parameter file, as chosen by you).  Within the
  parameter file, the ``batchno'' parameter specifies the current
  batch number. 
\end{enumerate}


\section{Controlling the simulation}
Once you have a running simulation, you may connect to it, to analyse
and debug it, and to display plots of any data you are interested in.
To do this, use the daspctrl.py GUI (which has a separate manual).  After the GUI has started, you
should first connect the simulation,
giving the hostname and port number that you wish to connect to.  By
default, the simulation processes will be listening on port number
9000 plus their MPI rank (so for a four process simulation you should
connect to ports 9000, 9001, 9002 and 9003 on the respective nodes on
which the processes are running).  If for any reason, the simulation
process was unable to open a listening port at this port number (for
example, someone else was running a simulation which had taken this
port), it will increase the port number by the total number of
processes running in your simulation (four in this case) and try
again.  This will repeat until it is able to open a port
successfully.  You should then click the ``Get plots'' button which
will interrogate the running simulation to find out what modules are
present, and provide buttons for control of the simulation, for
example opening and closing the loop, restarting a science exposure,
etc.  

Further details about the simulation control GUI are given in
\citet{simctrlgui}

The command line programme daspanalyse.py can also be used to
interrogate the simulation.  By default, with no arguments, it will
connect to a simulation on port 9000 and print out the PSF parameters
(Strehl ratio etc.) of all science objects within that simulation.
Other options are also available.  This provides a handy way to check
simulation progress.



\subsection{Visualisation}
Within the daspctrl.py GUI, it is possible to visualise aspects of the
simulation.  Once you have connected, click the ``Get plots'' button.
This will interrogate the running simulation to find out what modules
are present, and within these, what can be visualised.  Then, you can
select the relevant module, and display the relevant plots that you
are interested in, for example DM shape, corrected phase, WFS images,
short exposure PSFs, long integration PSFs, etc.

\subsection{Short cuts}
There are several shortcuts or handy hints which can be implemented
when running a simulation.  The first is to use the batch facility in
cases where a parameter space has to be explored.  By setting up a
variable in the parameter file which depends on batch number, a number
of different simulations can then be run just by using a different
batch number (use --batchno=XXX on the command line).

When you wish to run a large number of simulations which require user
input, e.g.\ to poke the deformable mirror and hence compute a poke
matrix, this can become tedious.  The use of the
\texttt{initialCommand} method in the control object can then be used
to automate these inputs, by giving commands to the simulation which
can be executed after a given iteration.  So, for example, you might
use this to open the loop and start a poke at iteration zero.  Then at
a number of iterations later, you might close the loop and start the
real simulation.  This then means that the simulation can be run
without user intervention from start to finish.  Daspsetup.py and
daspbuilder.py take this approach.

Finally, if you wish to create some statistics from a number (e.g. 10)
of runs of the same simulation (so that you can for example find an
error estimate for the computed Strehl ratio etc), you can create a
simple bash script which will run all the simulations for you.  This
of course will only work if the simulation requires no user
intervention between start and finish.  



\pagebreak
\bibliography{references}

\printindex
\end{document}
