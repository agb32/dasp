\documentclass{article}
\usepackage{graphicx}
\usepackage{natbib}
\usepackage{graphicx}
\bibliographystyle{plainnat}
\usepackage{fancyheadings}
\usepackage{tabularx}
\usepackage{alltt, parskip, boxedminipage}
\usepackage{makeidx, multirow, longtable, tocbibind, amssymb}
%\usepackage{fullpage}
\makeindex
\usepackage[usenames]{color}
\definecolor{darkblue}{rgb}{0,0.05,0.35}

\usepackage[dvips, pagebackref, pdftitle={}, pdfcreator={epydoc 2.1}, bookmarks=true, bookmarksopen=false, pdfpagemode=UseOutlines, colorlinks=true, linkcolor=black, anchorcolor=black, citecolor=black, filecolor=black, menucolor=black, pagecolor=black, urlcolor=darkblue]{hyperref}
\setlength{\textheight}{21.5cm}
\setlength{\textwidth}{18cm}
\setlength{\hoffset}{-3.0cm}
\setlength{\footskip}{1.5cm}
\setlength{\headsep}{0.5cm}
\setlength{\voffset}{-2.5cm}
\newlength{\BCL} % base class length, for base trees.

\usepackage{everyshi}
 \makeatletter
 \let\totalpages\relax
 \newcounter{mypage}
 \EveryShipout{\stepcounter{mypage}}
 \AtEndDocument{\clearpage
    \immediate\write\@auxout{%
     \string\gdef\string\totalpages{\themypage}}}
 \makeatother

\newcommand{\daspfooter}{
\includegraphics[height=1cm]{durlogo.eps}
{\bf \large Centre for \AA dvanced Instrumentation}
}
\newcommand{\daspheaderl}{
\includegraphics[height=1cm]{durlogo.eps}
\begin{tabularx}{9cm}{c}
\Large {\bf \dasptitle} \\ \large {\bf(\daspdoctype)}\\
\rightmark\hspace{0.1cm}
\end{tabularx}
\vfill
}
\newcommand{\daspheaderc}{
}
\newcommand{\daspheaderr}{
\begin{tabular}{|l|l|}\hline
Doc. number: & \daspdocno \\ \hline
Release date: & \daspreleasedate \\ \hline
Issue number: & \daspissue \\ \hline
Page number: & Page \thepage \ of \totalpages \\ \hline
Author(s) & \daspauthorname \\ \hline
\end{tabular}


}

\pagestyle{fancy}
\cfoot[]{}
\lfoot[\daspfooter]{\daspfooter}
\lhead[\daspheaderl]{\daspheaderl}
\chead[\daspheaderc]{\daspheaderc}
\rhead[\daspheaderr]{\daspheaderr}
\renewcommand{\sectionmark}[1]{\markboth{#1}{#1}}

\newenvironment{Ventry}[1]%
  {\begin{list}{}{%
    \renewcommand{\makelabel}[1]{\texttt{##1:}\hfil}%
    \settowidth{\labelwidth}{\texttt{#1:}}%
    \setlength{\leftmargin}{\labelsep}%
    \addtolength{\leftmargin}{\labelwidth}}}%
  {\end{list}}


\begin{document}
\include{title}
\renewcommand{\dasptitle}{The cell processor for AO simulation}
\renewcommand{\daspdocno}{AOSIM-CEL-UoD-001}
\thispagestyle{empty}
%This next command provides the CVS tag.  If you want a cvs tag on
%your document, add the following line at the start of the document,
%after replacing the & signs with dollar signs...
%\newcommand{\cvsID}{& &Id& (CVS)&}
\providecommand{\cvsID}{CVS ID not provided: document made on \today}

\begin{center}
\includegraphics{durlogo.eps}
\end{center}
\vspace{0.5cm}
\Huge
\begin{center}
\daspproject\\
\end{center}
\Large
\vspace{1cm}


{\bf 
\begin{tabular}{ll}
Document title: & \dasptitle \vspace{0.5cm}\\ 

Documentation number: & \daspdocno \vspace{0.5cm}\\ 

Document type: & \daspdoctype \vspace{0.5cm}\\ 

Issue number:& \daspissue \vspace{0.5cm}\\ 

Release date: & \daspreleasedate \\ 

\end{tabular}
}

\normalsize
\vfill

\begin{tabular}{|l|l|l|p{5cm}|}
\hline
Document & \daspauthorname & Signature &\\
prepared by & \daspauthortype & and date &\\ \hline
Document & \daspapprovername & Signature &\\
approved by & \daspapprovertype & and date &\\ \hline
Document & \daspreleasername & Signature &\\
released by & \daspreleasertype & and date &\\ \hline
Document & \daspreviewername & Signature &\\
reviewed by & \daspreviewertype & and date &\\ \hline
\end{tabular}

\small
%\begin{alltt}
\cvsID
%\end{alltt}
\normalsize
%\lfoot[\daspfooter]{\daspfooter}
%\lhead[\daspheaderl]{\daspheaderl}
%\chead[\daspheaderc]{\daspheaderc}
%\rhead[\daspheaderr]{\daspheaderr}
%\renewcommand{\sectionmark}[1]{\markboth{#1}{#1}}

\pagebreak
a
\vspace{2cm}
\begin{center}
\Large
{\bf Change record\\ \vspace{1cm}}
\normalsize
\daspchangerecord
\end{center}
\vspace{2cm}

\begin{center}
\Large
{\bf Notification list\\ \vspace{1cm}}
\normalsize
\daspnotificationlist
\end{center}

\pagebreak

\begin{center}
\Large
{\bf Acronyms and abbreviations\\ \vspace{1cm}}
\normalsize
\daspabbreviations
\end{center}

\pagebreak

\begin{center}
\Large
{\bf Applicable documents\\ \vspace{1cm}}
\normalsize

\daspapplicabledocs
\end{center}
\vspace{2cm}

\begin{center}
\Large
{\bf Reference documents \\ \vspace{1cm}}
\normalsize

\dasprefdocs
\end{center}

\pagebreak
\tableofcontents
\pagebreak

\section{Introduction}
This document is intended to give details about simulation code using
the cell processor, including work to be done, and completed work.  It
will be useful for simulation users, and cell processor developers.
Further information about the AO simulation package can be found in
\citet{overview}.  In this document the use of SPU and SPE will be
used interchangable (Synegetic Processing Element, Engine or Unit).

\section{Code porting}
Initially, two algorithms should be ported to the cell processor, the
science image generation algorithm, and the wavefront sensor
algorithm.  Both of these algorithms are implemented in FPGAs, so a
comparison of execution time, and development time will be
interesting.

\subsection{Science image algorithm}
The science image algorithm to be implemented in the cell processor
can be found in the \texttt{util/sci.py} file, method
\texttt{computeShortExposurePSF}.  The important code for this
algorithm is as described, the algorithm
\texttt{computeShortExposurePSF} below:

\begin{verbatim}
class psf:
    def __init__(self,npup,fftsize):
        """npup is the number of phase pixels across the telescope pupil
        fftsize is the number of pixels to use for FFT.  Typically
        this will be 2*npup, maybe rounded up to the nearest power of 2.
        The maximum size of npup is likely to be 4096, though could be
        reduced to 1024 if absolutely necessary.
        """
        self.fftsize=fftsize
        self.npup=npup
        self.pupilmap=util.tel.Pupil(npup,npup/2,npup/8).fn
        self.pupilAmplitude=Numeric.zeros((fftsize,fftsize),Numeric.Float32)
        self.focusAmplitude=self.pupilAmplitude# for an inplace transformation
        self.tempImg=Numeric.zeros((fftsize,fftsize),Numeric.Float32)
        self.instImg=Numeric.zeros((fftsize,fftsize),Numeric.Float32)

    def computeShortExposurePSF(self,phasemap,fftsize,pupilmap):
        npup=phasemap.shape[0]
        self.pupilAmplitude[npup:]=0.#note, pupilAmplitude.shape=(fftsize,fftsize)
        self.pupilAmplitude[:npup,npup:]=0.
        self.pupilAmplitude.real[:npup,:npup]=pupilmap*Numeric.cos(phasemap)
        self.pupilAmplitude.imag[:npup,:npup]=pupilmap*Numeric.sin(phasemap)

        self.focusAmplitude[:]=FFT2D(self.pupilAmplitude)#this might be an
        #inplace transformation to save memory
    
        self.tempImg[:]=Numeric.absolute(self.focusAmplitude)**2

        util.flip.fliparray2(self.tempImg,self.instImg)#function from aosim/util/
        self.instImg/=Numeric.sum(Numeric.sum(self.instImg))
        return self.instImg
\end{verbatim}

I think that the SPE processors are best viewed as application
accelerators for the PPC processor, rather than as separate
multi-processors.  The \texttt{FFTW} library has been ported to the
cell platform, and so this can make work a lot easier.  However, there
are still a number of considerations.

A two dimensional FFT is basically a large number of 1D FFTs in one
direction, and then a large number of 1D FFTs on the resulting data in
the orthogonal direction.  

How can best performance be achieved?  The simplest way of
implementing this is to divide the work into stages, and then
implement each of these stages in the SPEs, i.e.\ the first stage
would be to take the cos and sine of the data.  The next stage, would
be implemented as one, doing the FFT, letting \texttt{FFTW} handle the
parallelisation.  Another stage would be the squaring, another would
be the reordering, and a final stage would be the normalisation of the
data.

However, this involves passing data between the PPC and SPEs a lot,
and so may suffer a bottleneck here.  It may be better to first do the
cos and sine of the data, followed by one dimensional FFTs within the
SPEs, before returning data to the PPC.  This data can then be
transposed and passed back to the SPEs for the second one dimensional
FFT (completing the 2D FFT), and the squaring and reordering of the
data can the be done still in the SPEs, as well as summing the data
within the SPE, before the data is passed back to the PPC.  The
normalisation factor is then the sum of factors from each SPE, and the
data can then be passed back to the SPEs to perform this
normalisation.  Using this scenario, data is passed to and from the
SPEs only three times, compared with six stages for the more
simplistic approach.  This way is therefore likely to give better
performance.

The size of the FFT can be assumed to be at most $4096\times4096$
phase elements, oversampled for the FFT to $8192\times8192$ elements.
If necessary, this could be restricted further to $1024\times1024$
phase elements oversampled to $2048\times2048$ elements for the FFT
(which is what the FPGA can manage).


\section{General approach}
If possible, the same code should be usable on cell and conventional
platforms.  This could be achieved using either compiler flags, or
probably better, by creating a dummy libspe on other platforms that
can then be called by the code, but does nothing exotic (e.g.\ it
might just run code).  This is probably the better approach, as it
would make the code more readable.  Switches in the makefile would
then be used to compile using spu-gcc and embedspu or gcc as
appropriate.  This will be the long term aim.  Even on the PS3, the
dummy libspe should be used if the code is just to be tested in the
PPC, not using the SPEs.  A define can be used to determine which
libspe header file should be used.  Or something!

Probably the best way to start is to develop this algorithm for the
PS3, and benchmark it.  Then, depending on the results, it may or may
not be worth carrying on.  If it is worth carrying on, then a more
sustainable systems engineering approach should be used, to integrate
the code fully with the simulation.  In the first instance, the code
can be C only, no need to include hooks to allow it to be called by
Python.

\section{Notes}
The PS3 uses a gateway on aipc52 so that it can reach the cvs server.
When setting up your cvs password etc, see the aliases that user ali
has set in his .bashrc file.  This allows to you to do:
\begin{verbatim}
cvstunnel (enter tunnel password)
cvslogin (enter your password)
cvs checkout aosim
\end{verbatim}
to retrieve the simulation.  If you get stuck, try asking Ali or Gary
McCallum.

\subsection{Starting point}
Initially, try to just get some code working (follow the IBM
examples) in the SPEs.  Then try to get some more CPU intensive code
working, for example a vector dot product or an array operation or
something, and compare with the time taken in the PPC with the time
when using the SPE.  After this, start work on the algorithm
described above.


\bibliography{references}
\printindex
\end{document}
