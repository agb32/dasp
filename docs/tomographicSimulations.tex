\documentclass{article}
\usepackage{graphicx}
\usepackage{natbib}
\usepackage{graphicx}
\bibliographystyle{plainnat}
\usepackage{fancyheadings}
\usepackage{tabularx}
\usepackage{alltt, parskip, boxedminipage}
\usepackage{makeidx, multirow, longtable, tocbibind, amssymb}
%\usepackage{fullpage}
\makeindex
\usepackage[usenames]{color}
\definecolor{darkblue}{rgb}{0,0.05,0.35}

\usepackage[dvips, pagebackref, pdftitle={}, pdfcreator={epydoc 2.1}, bookmarks=true, bookmarksopen=false, pdfpagemode=UseOutlines, colorlinks=true, linkcolor=black, anchorcolor=black, citecolor=black, filecolor=black, menucolor=black, pagecolor=black, urlcolor=darkblue]{hyperref}
\setlength{\textheight}{21.5cm}
\setlength{\textwidth}{18cm}
\setlength{\hoffset}{-3.0cm}
\setlength{\footskip}{1.5cm}
\setlength{\headsep}{2.5cm}
\setlength{\voffset}{-2.5cm}
\newlength{\BCL} % base class length, for base trees.

\usepackage{everyshi}
 \makeatletter
 \let\totalpages\relax
 \newcounter{mypage}
 \EveryShipout{\stepcounter{mypage}}
 \AtEndDocument{\clearpage
    \immediate\write\@auxout{%
     \string\gdef\string\totalpages{\themypage}}}
 \makeatother

\newcommand{\esofooter}{
\includegraphics[height=1cm]{durlogo.eps}
{\bf \large Centre for \AA dvanced Instrumentation}
}
\newcommand{\esoheaderl}{
\includegraphics[height=1cm]{durlogo.eps}
\begin{tabularx}{9cm}{c}
\Large {\bf \esotitle} \\ \large {\bf(\esodoctype)}\\
\rightmark\hspace{0.1cm}
\end{tabularx}
\vfill
}
\newcommand{\esoheaderc}{
}
\newcommand{\esoheaderr}{
\begin{tabular}{|l|l|}\hline
Doc. number: & \esodocno \\ \hline
Release date: & \esoreleasedate \\ \hline
Issue number: & \esoissue \\ \hline
Page number: & Page \thepage \ of \totalpages \\ \hline
Author(s) & \esoauthorname \\ \hline
\end{tabular}


}

\pagestyle{fancy}
\cfoot[]{}
\lfoot[\esofooter]{\esofooter}
\lhead[\esoheaderl]{\esoheaderl}
\chead[\esoheaderc]{\esoheaderc}
\rhead[\esoheaderr]{\esoheaderr}
\renewcommand{\sectionmark}[1]{\markboth{#1}{#1}}

\newenvironment{Ventry}[1]%
  {\begin{list}{}{%
    \renewcommand{\makelabel}[1]{\texttt{##1:}\hfil}%
    \settowidth{\labelwidth}{\texttt{#1:}}%
    \setlength{\leftmargin}{\labelsep}%
    \addtolength{\leftmargin}{\labelwidth}}}%
  {\end{list}}


\begin{document}
\include{title}
\renewcommand{\esotitle}{Preparing tomographic simulations}
\renewcommand{\esodocno}{AOSIM-TOM-UoD-001}
\thispagestyle{empty}
%This next command provides the CVS tag.  If you want a cvs tag on
%your document, add the following line at the start of the document,
%after replacing the & signs with dollar signs...
%\newcommand{\cvsID}{& &Id& (CVS)&}
\providecommand{\cvsID}{CVS ID not provided: document made on \today}

\begin{center}
\includegraphics{durlogo.eps}
\end{center}
\vspace{0.5cm}
\Huge
\begin{center}
\esoproject\\
\end{center}
\Large
\vspace{1cm}


{\bf 
\begin{tabular}{ll}
Document title: & \esotitle \vspace{0.5cm}\\ 

Documentation number: & \esodocno \vspace{0.5cm}\\ 

Document type: & \esodoctype \vspace{0.5cm}\\ 

Issue number:& \esoissue \vspace{0.5cm}\\ 

Release date: & \esoreleasedate \\ 

\end{tabular}
}

\normalsize
\vfill

\begin{tabular}{|l|l|l|p{5cm}|}
\hline
Document & \esoauthorname & Signature &\\
prepared by & \esoauthortype & and date &\\ \hline
Document & \esoapprovername & Signature &\\
approved by & \esoapprovertype & and date &\\ \hline
Document & \esoreleasername & Signature &\\
released by & \esoreleasertype & and date &\\ \hline
Document & \esoreviewername & Signature &\\
reviewed by & \esoreviewertype & and date &\\ \hline
\end{tabular}

\small
%\begin{alltt}
\cvsID
%\end{alltt}
\normalsize
%\lfoot[\esofooter]{\esofooter}
%\lhead[\esoheaderl]{\esoheaderl}
%\chead[\esoheaderc]{\esoheaderc}
%\rhead[\esoheaderr]{\esoheaderr}
%\renewcommand{\sectionmark}[1]{\markboth{#1}{#1}}

\pagebreak



\begin{center}
\Large
{\bf Change record\\ \vspace{1cm}}
\normalsize
\esochangerecord
\end{center}
\vspace{2cm}

\begin{center}
\Large
{\bf Notification list\\ \vspace{1cm}}
\normalsize
\esonotificationlist
\end{center}

\pagebreak

\begin{center}
\Large
{\bf Acronyms and abbreviations\\ \vspace{1cm}}
\normalsize
\esoabbreviations
\end{center}

\pagebreak

\begin{center}
\Large
{\bf Applicable documents\\ \vspace{1cm}}
\normalsize

\esoapplicabledocs
\end{center}
\vspace{2cm}

\begin{center}
\Large
{\bf Reference documents \\ \vspace{1cm}}
\normalsize

\esorefdocs
\end{center}

\pagebreak
\tableofcontents
\pagebreak

\section{Introduction}
This document is intended to give details about specifying tomographic
simulations.  It should be useful for simulation users and developers.
Further information about the AO simulation package can be found in
\citet{overview}.

\section{Specifying tomographic simulations}
When specifying a tomographic simulation, a clear and coherent method
is needed to specify all the simulation parameters.

There will be a number of atmospheric layers, a number of source
directions, a number of DMs conjugated to different heights, viewing
the different sources, a number of wavefront sensors (including LGS
with different spot elongation patterns), one or more reconstructors,
science targets, and various other bits and pieces.

There needs to be an intuitive way to store all this information in
the parameter file in a way that ensures that if one parameter is
changed (e.g.\ one source direction), then all simulation objects are
aware of this change.

For example, if a source direction is changed, each atmospheric screen
module needs to know so that the size of the screen can be adjusted
accordingly.  The atmospheric pupil phase modules need to know so that
the size of the screens are known.  The DM modules need to know so
that they know how much overlap the source pupil will have with other
pupils on this DM.  The wavefront sensors need to know so that any
spot elongation patterns can be updated, and the reconstructor also
needs to know so that e.g.\ the theoretical poke matrix can be created
correctly.  

The nature of the simulation means that each source direction for each
DM is specified as a separate object (though resource sharing can
occur if necessary).  Therefore, each DM object needs to know which
physical DM it is (conjugate height, actuator count, etc.), and which
source direction it corresponds to.  A sensible scheme therefore
required to map a specific DM object to its respective DM information
and source information.  

The reconstructor may also need to know which source directions the
wavefront signals are from, and the order of each wavefront sensor.
It cannot simply assume that all source directions are used for
wavefront sensing.  A method to detail this is also therefore
required.  

\section{Proposed solution}
The proposed solution for this is as follows.  An atmospheric geometry
object is first specified (in the simulation parameter file).  This
object has a dictionary of atmospheric layer details, a list of
source details, and a few telescope details (diameter, number of phase
pixels to use).  The layer dictionary has the layer ID string as a key
(e.g.\ 2000m), and then a tuple of (height, wind direction, speed,
strength, initialisation seed).  The source list is a list of tuples of
(idstr, $\theta$,$\phi$,altitude, nsubx).  Here, altitude is -1 for a source at
infinity, or $>$0 for a LGS.  nsubx is the number of sub-apertures or
None if the source is a science object only.  The idstr is the
identification string used for this source direction, for example
``offaxis'', ``source1'' etc.

A DM object is also specified.  This is passed a list of tuples, one
tuple for each DM object (note, there are probably many DM objects for
each physical DM).  Each tuple contains (DM ID string, conjugate
height, source ID string, number of actuators, field of view, actuator
coupling).  Here, the source ID string is the ID string used to select
the source direction from the atmospheric geometry object.  The DM ID
string is the ID string for the DM object to which this tuple applies.
Note, a dictionary is not used here because ordering of the DM objects
is important in the reconstructor.  The field of view can be None, in
which case the minimum field that will fit all sources is chosen.  The
actuator coupling can be None, in which case a default of 0.1 is
selected.

By using these two objects, it should then be possible for all the
simulation objects to be defined in a way that makes them aware of the
geometries of the system.

\subsection{Object identification}
Each simulation object can be given an ID string, which helps to
specify which unique parameters are extracted from the parameter file.
These ID strings can be anything, however here, a fairly sensible
scheme for naming is given.

Atmospheric layer objects (e.g.\ infScrn) should be given an idstr
that mentions the layer height.  Personally, I would use e.g.\
``2000m'', ``0m'' etc.

Source direction object (e.g.\ infAtmos) should be given an idstr that
specifically mentions them.  This could be simple or more complex.
For example, you might choose ``a'', ``b'', ``c'', etc.  Or you might
choose ``t0p0'', ``t10p0'', ``t10p180'' for sources at $(\theta=0,
\phi=0)$, $(\theta=10, \phi=0)$, $(\theta=10, \phi=180)$.  Really,
just something that helps you.

DM objects, (e.g.\ el\_dm, xinterp\_dm) can then be named as a
combination of their conjugate height and the source direction, e.g.\
``1500ma'' or ``0mt0p0'' or similar.

By following this naming convention, the simulation should maintain
some clarity.

The tomoRecon object uses the keys in the parent dictionary to select
the DM objects, so these must be named appropriately.  Note, that if
using the same scheme with the xinterp\_recon object (i.e.\ if you are
doing a classical AO problem, 1 guide star etc.) then the parent keys
expected will start with ``cent''.  Therefore your DM objects should
be named ``cent''+something.

\section{Reconstruction}

Several different reconstruction techniques have been attempted, all
hoping to solve the problem $Ax=b$ for $x$ where $A$ is the poke
matrix (or equivalent) and $b$ are the centroid measurements.  

A FDPCG reconstructor has been implemented and runs, but does not
successfully close the loop, as the problem blows up.

By multiplying the equation by $A^T$, you are then trying to solve
$A^TAx=A^Tb$, and the $A^TA$ matrix is square, and so can be solved
for using LU decomposition/Gaussian elimination.  However this
approach does not seem to work.

The other approach that has been successfully tried is SVD to create a
pseudo-inverse for $A$.  This works successfully.  Some investigations
into sparsitying the pseudo-inverse have also been carried out, and
the performance doesn't seem to degrade if only $2.5\%$ of information
is used in the pseudo-inverse for some guide-star configurations.
This therefore could make this a possibility for large simulations,
e.g.\ EAGLE, where the pseudo-inverse would be 40~GB in size.  By
sparsifying, it could be reduced to 1~GB, or possibly even less.  

Even if only $1\%$ of the pseudo-inverse is used, correction is still
obtained, though at a lower level than when more of this matrix are
used.  To sparsify in these tests, every value (magnitude) below a
certain level was simply set to zero.

A MAP reconstructor is also in the process of being ported from
classical AO to MCAO etc.  

\subsection{Wavelengths}
The wavelength at which the reconstructor reconstructs is not clearly
defined or particularly easy to understand.  In the case where all WFS
are at the same wavelength, it will be at this wavelength. 
The dmInfo object specified in the parameter file can be given a
reconLam parameter, which is the wavelength at which the reconstructor
values are applied to.  The mirror surface will then be assumed to be
valid for thsi wavelength.  Other source directions (other
wavelengths) then scale as appropriate on output.

In more detail, if this reconLam isnt specified, the DM will get its
wavelength from the idstr, from which it will obtain the source
direction ID, and use this in atmosGeom to get the wavelength of this
direction.  The real question then is what wavelength are the
reconstructor actuator values for.  The reconstructor wavelength is
specified by atmosGeom, using a source ID which is obtained from the
config file, with a default value equal to the first one in the DM
list.  This is then used to compute the pixel scale, and to compute
the Noll matrix scaling.  The pixel scale itself is used only when
computing the theoretical noise covariances for MAP.

So, what actually happens in tomographic situations?  When you are
creating a poke matrix, the DM commands are sent to the mirror.  The
WFS that is used by the first DM in the resource sharing object will
then see pokes equal to those sent by the reconstructor, while other
WFS at different wavelengths will see pokes scaled by the wavelength.
The reconstructor will then have centroids measured at different
wavelengths.  It will compute the reconstruction matrix, suited to the
wavelength of the first DM.  It should be noted that if the
reconstruction matrix is computed using information about the
wavelength (eg for MAP), the components (phase covariance, noise
covariance) should be scaled correctly for wavelength (ie a
theoretical noise covariance should be differently for each wavefront
sensor).  The phase covariance should be scaled for the primary
wavelength of each DM.  

One thing to note, is that in open-loop simulations, you will probably
need to have a DM object for which the output is not used.  This DM
will be the first of the resource sharing objects, and will then be
shaped by the reconstructor to the primary wavelength (which is that
for which it was poked).  Other DM resource sharers will then be
scaled accordingly.

So, in summary, a DM with same wavelength should be the first of a
resource sharing set in both the poking and (for open-loop
simulations) science simulations (which may mean a redundant DM object
is needed, ie the output is not used).  This DM must also be the one
listed first in dmObj list, unless the reconstuctor is given a
sourceID parameter.

\section{Tip-tilt}
The current way to do tip-tilt is to use a zdm mirror or a magicDM,
set to remove the first 3 modes (PTT).  Note, the actuator values may
need to be scaled by splitOutput so that the tip-tilt is of a
reasonable magnitude.  Also note that any LGS wavefront sensors should
not see the mirror even when poking (otherwise, they would think the
mirror gives them a tip-tilt too).



\section{Conclusion}
Hope that helped!


\bibliography{references}
\printindex
\end{document}



