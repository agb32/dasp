\documentclass{article}
\usepackage{graphicx}
\usepackage{natbib}
\usepackage{graphicx}
\bibliographystyle{plainnat}
\usepackage{fancyheadings}
\usepackage{tabularx}
\usepackage{alltt, parskip, boxedminipage}
\usepackage{makeidx, multirow, longtable, tocbibind, amssymb}
%\usepackage{fullpage}
\makeindex
\usepackage[usenames]{color}
\definecolor{darkblue}{rgb}{0,0.05,0.35}

\usepackage[dvips, pagebackref, pdftitle={}, pdfcreator={epydoc 2.1}, bookmarks=true, bookmarksopen=false, pdfpagemode=UseOutlines, colorlinks=true, linkcolor=black, anchorcolor=black, citecolor=black, filecolor=black, menucolor=black, pagecolor=black, urlcolor=darkblue]{hyperref}
\setlength{\textheight}{21.5cm}
\setlength{\textwidth}{18cm}
\setlength{\hoffset}{-3.0cm}
\setlength{\footskip}{1.5cm}
\setlength{\headsep}{2.5cm}
\setlength{\voffset}{-2.5cm}
\newlength{\BCL} % base class length, for base trees.

\usepackage{everyshi}
 \makeatletter
 \let\totalpages\relax
 \newcounter{mypage}
 \EveryShipout{\stepcounter{mypage}}
 \AtEndDocument{\clearpage
    \immediate\write\@auxout{%
     \string\gdef\string\totalpages{\themypage}}}
 \makeatother

\newcommand{\esofooter}{
\includegraphics[height=1cm]{durlogo.eps}
{\bf \large Centre for \AA dvanced Instrumentation}
}
\newcommand{\esoheaderl}{
\includegraphics[height=1cm]{durlogo.eps}
\begin{tabularx}{9cm}{c}
\Large {\bf \esotitle} \\ \large {\bf(\esodoctype)}\\
\rightmark\hspace{0.1cm}
\end{tabularx}
\vfill
}
\newcommand{\esoheaderc}{
}
\newcommand{\esoheaderr}{
\begin{tabular}{|l|l|}\hline
Doc. number: & \esodocno \\ \hline
Release date: & \esoreleasedate \\ \hline
Issue number: & \esoissue \\ \hline
Page number: & Page \thepage \ of \totalpages \\ \hline
Author(s) & \esoauthorname \\ \hline
\end{tabular}


}

\pagestyle{fancy}
\cfoot[]{}
\lfoot[\esofooter]{\esofooter}
\lhead[\esoheaderl]{\esoheaderl}
\chead[\esoheaderc]{\esoheaderc}
\rhead[\esoheaderr]{\esoheaderr}
\renewcommand{\sectionmark}[1]{\markboth{#1}{#1}}

\newenvironment{Ventry}[1]%
  {\begin{list}{}{%
    \renewcommand{\makelabel}[1]{\texttt{##1:}\hfil}%
    \settowidth{\labelwidth}{\texttt{#1:}}%
    \setlength{\leftmargin}{\labelsep}%
    \addtolength{\leftmargin}{\labelwidth}}}%
  {\end{list}}


\begin{document}
\include{simapititle}
\thispagestyle{empty}
%This next command provides the CVS tag.  If you want a cvs tag on
%your document, add the following line at the start of the document,
%after replacing the & signs with dollar signs...
%\newcommand{\cvsID}{& &Id& (CVS)&}
\providecommand{\cvsID}{CVS ID not provided: document made on \today}

\begin{center}
\includegraphics{durlogo.eps}
\end{center}
\vspace{0.5cm}
\Huge
\begin{center}
\esoproject\\
\end{center}
\Large
\vspace{1cm}


{\bf 
\begin{tabular}{ll}
Document title: & \esotitle \vspace{0.5cm}\\ 

Documentation number: & \esodocno \vspace{0.5cm}\\ 

Document type: & \esodoctype \vspace{0.5cm}\\ 

Issue number:& \esoissue \vspace{0.5cm}\\ 

Release date: & \esoreleasedate \\ 

\end{tabular}
}

\normalsize
\vfill

\begin{tabular}{|l|l|l|p{5cm}|}
\hline
Document & \esoauthorname & Signature &\\
prepared by & \esoauthortype & and date &\\ \hline
Document & \esoapprovername & Signature &\\
approved by & \esoapprovertype & and date &\\ \hline
Document & \esoreleasername & Signature &\\
released by & \esoreleasertype & and date &\\ \hline
Document & \esoreviewername & Signature &\\
reviewed by & \esoreviewertype & and date &\\ \hline
\end{tabular}

\small
%\begin{alltt}
\cvsID
%\end{alltt}
\normalsize
%\lfoot[\esofooter]{\esofooter}
%\lhead[\esoheaderl]{\esoheaderl}
%\chead[\esoheaderc]{\esoheaderc}
%\rhead[\esoheaderr]{\esoheaderr}
%\renewcommand{\sectionmark}[1]{\markboth{#1}{#1}}

\pagebreak



\begin{center}
\Large
{\bf Change record\\ \vspace{1cm}}
\normalsize
\esochangerecord
\end{center}
\vspace{2cm}

\begin{center}
\Large
{\bf Notification list\\ \vspace{1cm}}
\normalsize
\esonotificationlist
\end{center}

\pagebreak

\begin{center}
\Large
{\bf Acronyms and abbreviations\\ \vspace{1cm}}
\normalsize
\esoabbreviations
\end{center}

\pagebreak

\begin{center}
\Large
{\bf Applicable documents\\ \vspace{1cm}}
\normalsize

\esoapplicabledocs
\end{center}
\vspace{2cm}

\begin{center}
\Large
{\bf Reference documents \\ \vspace{1cm}}
\normalsize

\esorefdocs
\end{center}

\pagebreak
\tableofcontents
\pagebreak

\section{Scope}
This document describes the AO simulation python modules in detail,
including their functions and behaviours and parameters required.  The
AO simulation software package, was designed for the AO simulation
project funded by the rolling grant.  

This document is relevant to anyone who is involved with the
development of the AO simulation software.  For an overview of the
simulation, please see \citet{overview}.


\subsection{Overview}
he AO simulation package contains modules which are used to simulate
the various parts of the operation of a telescope with an AO system
installed.  The scope of these modules ranges from creating
atmospheric turbulence to simulation of the closed loop telescope
system.  Each module is designed to perform a specific simulation
task, and are connected together by the user in such a way that
coherent results will be obtained from the simulation.

These modules communicate with each other by passing data, either with
in their process, using shared memory, or using MPI message passing.
The simulation framework has been designed so that this method of
message passing is transparent to the user.  Various base and utility
modules will allow the simulation to operate correctly.



%Add includes here
\bibliography{references}
\printindex
\end{document}
