\documentclass{article}
\usepackage{graphicx}
\usepackage{natbib}
\usepackage{graphicx}
\bibliographystyle{plainnat}
\usepackage{fancyheadings}
\usepackage{tabularx}
\usepackage{alltt, parskip, boxedminipage}
\usepackage{makeidx, multirow, longtable, tocbibind, amssymb}
%\usepackage{fullpage}
\makeindex
\usepackage[usenames]{color}
\definecolor{darkblue}{rgb}{0,0.05,0.35}

\usepackage[dvips, pagebackref, pdftitle={}, pdfcreator={epydoc 2.1}, bookmarks=true, bookmarksopen=false, pdfpagemode=UseOutlines, colorlinks=true, linkcolor=black, anchorcolor=black, citecolor=black, filecolor=black, menucolor=black, pagecolor=black, urlcolor=darkblue]{hyperref}
\setlength{\textheight}{21.5cm}
\setlength{\textwidth}{18cm}
\setlength{\hoffset}{-3.0cm}
\setlength{\footskip}{1.5cm}
\setlength{\headsep}{0.5cm}
\setlength{\voffset}{-2.5cm}
\newlength{\BCL} % base class length, for base trees.

\usepackage{everyshi}
 \makeatletter
 \let\totalpages\relax
 \newcounter{mypage}
 \EveryShipout{\stepcounter{mypage}}
 \AtEndDocument{\clearpage
    \immediate\write\@auxout{%
     \string\gdef\string\totalpages{\themypage}}}
 \makeatother

\newcommand{\daspfooter}{
\includegraphics[height=1cm]{durlogo.eps}
{\bf \large Centre for \AA dvanced Instrumentation}
}
\newcommand{\daspheaderl}{
\includegraphics[height=1cm]{durlogo.eps}
\begin{tabularx}{9cm}{c}
\Large {\bf \dasptitle} \\ \large {\bf(\daspdoctype)}\\
\rightmark\hspace{0.1cm}
\end{tabularx}
\vfill
}
\newcommand{\daspheaderc}{
}
\newcommand{\daspheaderr}{
\begin{tabular}{|l|l|}\hline
Doc. number: & \daspdocno \\ \hline
Release date: & \daspreleasedate \\ \hline
Issue number: & \daspissue \\ \hline
Page number: & Page \thepage \ of \totalpages \\ \hline
Author(s) & \daspauthorname \\ \hline
\end{tabular}


}

\pagestyle{fancy}
\cfoot[]{}
\lfoot[\daspfooter]{\daspfooter}
\lhead[\daspheaderl]{\daspheaderl}
\chead[\daspheaderc]{\daspheaderc}
\rhead[\daspheaderr]{\daspheaderr}
\renewcommand{\sectionmark}[1]{\markboth{#1}{#1}}

\newenvironment{Ventry}[1]%
  {\begin{list}{}{%
    \renewcommand{\makelabel}[1]{\texttt{##1:}\hfil}%
    \settowidth{\labelwidth}{\texttt{#1:}}%
    \setlength{\leftmargin}{\labelsep}%
    \addtolength{\leftmargin}{\labelwidth}}}%
  {\end{list}}


\begin{document}
\include{simnewstitle}
\thispagestyle{empty}
%This next command provides the CVS tag.  If you want a cvs tag on
%your document, add the following line at the start of the document,
%after replacing the & signs with dollar signs...
%\newcommand{\cvsID}{& &Id& (CVS)&}
\providecommand{\cvsID}{CVS ID not provided: document made on \today}

\begin{center}
\includegraphics{durlogo.eps}
\end{center}
\vspace{0.5cm}
\Huge
\begin{center}
\daspproject\\
\end{center}
\Large
\vspace{1cm}


{\bf 
\begin{tabular}{ll}
Document title: & \dasptitle \vspace{0.5cm}\\ 

Documentation number: & \daspdocno \vspace{0.5cm}\\ 

Document type: & \daspdoctype \vspace{0.5cm}\\ 

Issue number:& \daspissue \vspace{0.5cm}\\ 

Release date: & \daspreleasedate \\ 

\end{tabular}
}

\normalsize
\vfill

\begin{tabular}{|l|l|l|p{5cm}|}
\hline
Document & \daspauthorname & Signature &\\
prepared by & \daspauthortype & and date &\\ \hline
Document & \daspapprovername & Signature &\\
approved by & \daspapprovertype & and date &\\ \hline
Document & \daspreleasername & Signature &\\
released by & \daspreleasertype & and date &\\ \hline
Document & \daspreviewername & Signature &\\
reviewed by & \daspreviewertype & and date &\\ \hline
\end{tabular}

\small
%\begin{alltt}
\cvsID
%\end{alltt}
\normalsize
%\lfoot[\daspfooter]{\daspfooter}
%\lhead[\daspheaderl]{\daspheaderl}
%\chead[\daspheaderc]{\daspheaderc}
%\rhead[\daspheaderr]{\daspheaderr}
%\renewcommand{\sectionmark}[1]{\markboth{#1}{#1}}

\pagebreak
a
\vspace{2cm}
\begin{center}
\Large
{\bf Change record\\ \vspace{1cm}}
\normalsize
\daspchangerecord
\end{center}
\vspace{2cm}

\begin{center}
\Large
{\bf Notification list\\ \vspace{1cm}}
\normalsize
\daspnotificationlist
\end{center}

\pagebreak

\begin{center}
\Large
{\bf Acronyms and abbreviations\\ \vspace{1cm}}
\normalsize
\daspabbreviations
\end{center}

\pagebreak

\begin{center}
\Large
{\bf Applicable documents\\ \vspace{1cm}}
\normalsize

\daspapplicabledocs
\end{center}
\vspace{2cm}

\begin{center}
\Large
{\bf Reference documents \\ \vspace{1cm}}
\normalsize

\dasprefdocs
\end{center}

\pagebreak
\tableofcontents
\pagebreak

\section{Introduction}
This document is intended to be a place where updates to the AO
simulation can be placed, so that people know where to look if they
wish to check whether anything has changed or been updated.  Further
information about the AO simulation package can be found in
\citet{overview}.
\section{Added 1st May 2008}
Have clarified the wavelength at which DMs are applied.  The dmInfo
object now has a reconLam parameter, at which the DM surface is
assumed to be, and different wavelengths are then scaled from this as
appropriate.  

\section{Added 25th April 2008}
Added a base/fifo.py to create arbitrary delays in feedback loops.
Also added an FFTW module for science, which performs significantly
faster than the previous ctypes/acml version.  This can use multiple
threads.  Also added hypergeometric functions in util, used for phase
variances for non-Kolmogorov turbulence.

\section{Added 15th April 2008}
Added util/blockMatrix.py for creating a block matrix object, useful
for tomographic simulations, where phase covariances etc are in block
matrix form.

\section{Added 2nd April 2008}
Have added a way to transfer data between modules (same and different
MPI rank).  To do this, a module will call self.config.post(name,data)
where name is the name of the variable, e.g. ``test'' or
``covariances'' etc and data is the array to be transferred.  After
this has been called, all other modules can call
self.config.postGet(name) which returns the data.  This data should be
treated as read-only, though can be updated using self.config.post
again.  Only one object with same name can be stored.  This is
typically useful for transferring once-per-simulation data, for
example something computed by one module which is then needed by
another module, and typically will be initiated from the GUI.  It uses
the socket interface to pass the data, not MPI.

\section{Added 31st March 2008}
Magic DM module added which can remove a specified number of Zernike
modes from a wavefront (without needing a reconstructor).
Also added a proper Zernike DM module (zdm.py) which can be used with
xinterp\_recon.


\section{Added 20th March 2008}
Sorted out the wavelength variables.  These are now described in the
util.atmos.geom object, rather than separately in the param file.
Each optical path therefore has a given wavelength.  If you wish to
use an output at a different wavelength, then I suggest using a
splitOutput object which simply rescales the output.  The alternative
was to have all outputs defined at a given wavelength, e.g.\ 500nm.
However, this was deemed to be not as good as some time would be lost
rescaling for every input.

\section{Added 5th March 2008}
Simulation has been converted to use numpy as default.  Bugs may have
been caused, but I haven't found any yet.  Please use numpy as default
for anything you do.

\section{Added 28th Feb 2008}
A bug in Numeric when used with python2.5 has been found, i.e. in
Numeric arrays, [:] doesn't return the expected result, so [:,] should
be used instead.  This has been fixed in most (all?) places.

\section{Added 22nd Feb 2008}
Grouping now available in simsetup - makes setting up complex
tomographic simulations much simpler.
binCentroid module now available - simply bins output from a
centroider.


\section{Added 7th Feb 2008}
Tomographic reconstruction etc now possible.

\section{Added 12th Oct 2007}
Have implemented a sparse SVD reconstructor, which seems to do the
trick.  For a 2 DM system with 9 high order ($64\times64$) wavefront
sensors, the reconstruction matrix takes about 4 hours to generate, as
does the poke matrix.  After this, the simulation can then run.  For
EAGLE, it can be run in two stages, one to do wavefront sensing,
saving the DM commands (and atmospheres), and then another stage to
compute the PSF and performance.  

Two new base modules have been added, saveOutput.py and loadOutput.py,
which save an output each iteration or load a previously saved
output.

I believe the simulation is now virtually EAGLE ready, subject to some
LGS tests (and addition of a true minimum variance reconstructor!).

\section{Added 13th Sept 2007}
Have implemented a tomographic FDPCG reconstructor - it runs, but
doesn't close the loop.  Needs further work.  Still, its some
progress.

Have also implemented sparse matrix methods for tomographic poke
matricees.  Specifically, it is now possible to create a full EAGLE
poke matrix (non-sparse would be 40GB), and then to solve for phase.
To do this, multiply the poke matrix by itself transposed, and the
centroids by the poke matrix transposed.  Then have a problem in form
Ax=b where A is square and sparse, and so a sparse LU decomposition
can be done on A, meaning that x can then be solved for.  For an EAGLE
simulation, each reconstruction iteration should take between 0.5-1
seconds to complete.

\section{Added 31st July 2007}
wfscent module can now be run as a multi-threaded c module, which will
give performance increases for multi-core processors.  As a single
thread, it is about 3-5 times faster than the corresponding Python
code.  Improvements have been made to the SOR module, doubling the
performance.

\section{Added 19th July 2007}
The cell processor can now be used by the wfscent module, and is
approximately 6 times faster than a single Cray processor.  Currently,
scipy is not fully installed on the cell platform, and so the whole
simulation cannot be run here.  It is hoped that this will eventually
be fixed.

\section{Added Apr 11th 2007}
The el\_dm now uses resource sharing, to some extent.  Sharing is
allowed between objects that represent the same physical DM (ie
different wavelengths and source directions).  The conjugate height
must be the same.  Also, xinterp\_dm uses resource sharing, again for
a physical DM.

\section{Added Mar 30th 2007}
FDPCG reconstructor now works reliably (use xinterp\_recon object).
This is an $O(n\log n)$ reconstructor.  Unfortunately doesn't do as
well as SOR at low light levels, I think.

The xinterp\_dm module can now accept the el\_recon SOR/Zernike
reconstructor, as well as the xinterp\_recon reconstructor (zernike,
MAP or FDPCG).

\section{Added Mar 21st 2007}
The science module now has FPGA support (somewhere between 10-100
times faster than software, depending on pupil size), and can also bin
the pupil if requested.

Resource sharing works for science, infAtmos wfscent modules now.  It
should now be possible to put together a huge simulation!

\section{Added Feb 22nd 2007}
There are a number of new features that have been slowly added to the
simulation over the past few months.
\begin{enumerate}
\item simsetup GUI - allows graphical setup of a simulation.  This is
  now the recommended way to setup a simulation
\item Improvements to simctrl GUI - command history (try right
  clicking on the command text entry), ability to load
  plot information from the simulation.
\item Resource sharing - a new way of sharing resources between
  similar science modules.  Currently implemented in the science
  module.  Now means that one science module can do the work of many,
  re-using memory.  So it is possible to have a large number of ELT
  scale science modules running on a single mode.  Once the FPGA code
  for science has been completed (currently in debugging stage), this
  will allow rapid science image calculation for many science objects.
  The eventual aim is to implement resource sharing for wfscent, atmos
  and DM modules as well.  This will greatly reduce the memory
  requirements for large simulations.
\item aosim command - if you create a simulation using the simsetup
  GUI, you can now run it using \texttt{aosim mysim.py} where mysim is
  the name of your simulation file.
\item Connection between simsetup and paramgui - allows you to edit
  parameters as you set up the simulation.
\item I think thats all for now...
\end{enumerate}

\section{TODO}
There are a number of things that need to be done.
\begin{enumerate}
\item Stop using Numeric, and move everything to numpy.  This will be
  a huge task!
\item Have a tool which will create a default parameter file for you,
  given a few input parameters, e.g. telescope size, number of
  atmospheric layers etc, choosing sensible defaults for unset
  parameters. 
\end{enumerate}


\bibliography{references}
\printindex
\end{document}
